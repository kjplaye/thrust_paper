\documentclass[12pt,a4paper]{article}
\usepackage[width=.75\textwidth]{caption}
\usepackage{graphicx}
\usepackage{authblk}
\usepackage{amsmath}
\usepackage{amsfonts}
\usepackage{braket}
\usepackage{epigraph}
%\usepackage{mathrsfs}
\usepackage[mathscr]{euscript}
\usepackage[top=2cm, bottom=2cm, left=2cm, right=2cm]{geometry}
\usepackage{fancyhdr}

\setlength{\epigraphwidth}{0.8\textwidth}

% \pagestyle{fancy}
\begin{document}

%title and author details
\title{Field Theoretic Thrust of an Accelerating Frame}
\author[1]{Kevin Player\footnote{kplayer@andrew.cmu.edu}}

\maketitle

\epigraph{The Unruh effect tells us that what we call particles is really just a matter of perspective.}{Lee Smolin}


\abstract{We consider the field theoretic picture where Unruh radiation was initially uncovered. We demonstrate how
thrust can serve as a more accurate and direct explanation of the Unruh effect and hence Hawing radiation.}

\section{Introduction}
Much of this note including notation and conventions are drawn directly from the overview \cite{Frodden}, from the original source \cite{unruh}, and from \cite{beisert}.  As in these references, we only consider the scalar free field which is enough for the Unruh effect to become manifest.

\section{The Unruh Effect from Field Theoretic Thrust}
We consider a field theoretic version of thrust, driving sources.  From this, we will derive the Unruh vacuum expectation.  


\subsection{Rindler and Minkowski Modes Review}
Let $\hbar$ = c = 1.

Consider a $(1,1)$-dimensional spacetime\footnote{The full $(1,3)$-dimensional case does not add anything to the argument, so we stick to the dimensions where the linear acceleration boosts are taking place ($t$ and $z$).} $(t,z)$ with coordinates $u = -t + z$ and $v = -t - z$.  Any function of purely $u$, $f(u)$, is made up of Minkowski modes that are constant on $v$.  That is they are given by a 1-dimensional Fourier expansion
\[
f(u) = \frac{1}{2\pi} \int{e^{i p_u u} \hat{f}(p_u) dp_u}.
\]
supported on $p_v = 0$.  The same is true for $v$, $g(v)$, $p_v$, and $\hat{g}(v)$.  The full 2-dimensional transform of combinations of $f$ and $g$ is a linear combination of $\hat{f}(p_u) \delta(p_v)$ and $\hat{g}(p_v) \delta(p_u)$.  These are supported on the massless shell --- ``the momentum light cone''.


\begin{figure}[h]
\centering
\includegraphics[scale=0.4]{rindler_w.png}
\caption{Rindler wedge $W$ on the right.}
\label{rindlerw}
\end{figure}


Let $W$ be the $z>|t|$ Rindler wedge with coordinates
\[
t = \frac{1}{a}e^{a\xi}\sinh{(a\eta)}
\]
\[
z = \frac{1}{a}e^{a\xi}\cosh{(a\eta)}
\]
where $a>0$ is an acceleration parameter (see Figure \ref{rindlerw}).  We call $u$ and $v$ ``event horizon'' coordinates since they are defined along the event-horizons of an accelerating observer in $W$.


\begin{figure}[h]
\centering
\includegraphics[scale=1.0]{massless_shell.png}
\caption{Massless shell is when $p_u$ = $p_v$ = 0}
\label{masslessshell}
\end{figure}


For wave number $k$ and positive frequency $\omega_k$ consider the functions
\[
h^{(u)}_k = \frac{e^{\frac{\pi \omega_k}{2a}} {(au)}^{\frac{i\omega_k}{a}}}{ \sqrt{2\sinh\left(\frac{\pi\omega_k}{a}\right)} }
\]
\[
h^{(v)}_k = \frac{e^{\frac{\pi \omega_k}{2a}} {(av)}^{\frac{i\omega_k}{a}}}{ \sqrt{2\sinh\left(\frac{\pi\omega_k}{a}\right)} }
\]
The scalar field in Rindler curved space has modes
\[
 r_k = \frac{1}{\sqrt{4 \pi \omega_k}} e^{-i(\omega_k \eta -k \xi)}
\]
which can be written as combinations of the ``event horizon'' modes
\[
e^\frac{\pi\omega_k}{2a} h^{(u)}_k - e^{-\frac{\pi\omega_k}{2a}} h^{(v)}_k  = \sqrt{2 \sinh \left({\frac{\pi\omega_k}{a}}\right)} r_k
\]
Unlike the Rindler mode, the event horizon modes extend analytically from $W$ to all of spacetime.


When we switch between $h_k$ and the usual Minkowski bases we find that the Minkowski vacuum has excitations visible in the Rindler frame.  More specifically, there is a positive particle number vacuum expectation of
\begin{equation}
\label{radeq}
\frac{1}{e^{\frac{2 \pi \omega_k}{a}}-1}
\end{equation}
which is usually interpreted as (Unruh) radiation \cite{unruh}.

\subsection{Fourier Transform of the Sources}
We will take several Fourier transforms to elucidate the various modes and set up for some
integrals.  Let $\phi$ be a scalar free field in the flat $(1,1)$-dimensional Minkowski spacetime.  We will consider $h^{(u)}_k$ and $h^{(v )}_k$ as driving $W$-event horizon sources
\begin{equation}
\label{ab}
\rho = \alpha h^{(u)}_k + \beta h^{(v)}_k.
\end{equation}
with positive real convex combination $(\alpha + \beta) = 1$.
The drivers $h^{(u)}_k$ and $h^{(v)}_k$ are functions $f(u)$ of the past $W$-horizon and $g(v)$ of the future $W$-horizon respectively.  Both generate excitations, which we identify with absorption and emission thrusts respectively.


The source can originate from a coupling term, $\rho \phi$, added to the Lagrangian.
\[
\mathscr{L}_{driven} = \mathscr{L}_{original} + \rho\phi 
\]
This leads to an inhomogeneous Klein-Gordon equation
\[
(\Box + m^2) \phi = \rho
\]
as presented in \cite{beisert}\footnote{In \cite{beisert} it is assumed that the source is only active for a finite amount of time.  We let $\rho$ be active for all time.  The argument in \cite{beisert} seem to be adaptable to $\rho$.}

We want to integrate $\rho$ on shell in momentum space, which for a massless source is the positive energy part of the massless shell.  The two positive energy ``horizons'' border $p_u <= 0$ and $p_v <= 0$, see Figure \ref{masslessshell}.  Proceeding to take the Fourier transform of the function $f(u)$\footnote{WLOG since $g(v)$ is of the same form.}, we drop $p_u$ to just $p$ for the time being to increase legibility.  We will continue to assume that $\omega_k$ and $a$ are positive. Define the kernel
\[
  A = e^{-i p u} (au)^\frac{i\omega_k}{a} du
\]
and then we have
\begin{equation}
\label{finalnorm}
  \hat{f}(\_u) =  \frac{e^{\frac{\pi \omega_k}{2a}}}{\sqrt{2 \sinh \left({\frac{\pi\omega_k}{a}}\right)}}  \int_{-\infty}^\infty A
\end{equation}
where $L=\int_{-\infty}^0 A$ and $R=\int_0^\infty A$ are the left and right sides of the total integral $L + R$.

We rewrite the integrals using a complex changes of variables, $s$ = $ipu$, and contour integrals.

\begin{figure}[h]
\centering
\includegraphics[scale=0.6]{contour.png}
\caption{Using contours with large radius we convert the $L$ integral that goes to $i\infty$, and the $R$ integral that goes to $-i\infty$, to integrals with $s$ going to real $\infty$.}
\label{fig:x cubed graph}
\end{figure}


The $L$ integral for real $p<0$ is
\[
\begin{split}
  L(p) & = -\int_0^{i\infty} \left(\frac{ias}{-p}\right)^\frac{i\omega_k}{a} \left(\frac{i}{-p}\right)ds \\
  & = \frac{-i}{a} \left(\frac{a}{-p}\right)^{\frac{i\omega_k}{a} + 1} \int_0^\infty \left(is\right) ^ \frac{i\omega_k}{a} e^{-s} ds \\
  & = \frac{-i}{a} \left(\frac{a}{-p}\right)^{\frac{i\omega_k}{a} + 1} \Gamma\left(\frac{i\omega_k}{a} + 1\right) e^{-\frac{\pi \omega_k}{2a}} \\
  & = \frac{1}{2} \Gamma\left(\frac{i\omega_k}{a} + 1\right) e^{-\frac{\pi \omega_k}{2a}} B(p)\\
\end{split}
\]
where
\[
B(p) = \frac{-2i}{a} \left(\frac{a}{-p}\right)^{\frac{i\omega_k}{a} + 1} 
\]
is as shown.  This is using a large radius contour which rotates the endpoint 90 degrees clockwise.

The same calculation for $R$ is done using a counter-clockwise contour this time.
\[
\begin{split}
  R(p) = \frac{1}{2}\Gamma\left(\frac{i\omega_k}{a} + 1\right) e^{-\frac{\pi \omega_k}{2a}} B(p)
\end{split}
\]

We get back to $h_k^{(u)}$ and apply the normalization from equation (\ref{finalnorm})
\[
\hat{h_k}^{(u)}(p_u) = \frac{e^{\frac{\pi \omega_k}{2a}}}{\sqrt{2 \sinh \left({\frac{\pi\omega_k}{a}}\right)}}  ( L(p_u) + R(p_u) )
\]
So
\begin{equation}
\label{fourier}
\begin{split}
\hat{h_k}^{(u)}(p_u) & = \frac{\Gamma\left(\frac{i\omega_k}{a} + 1\right)}{\sqrt{2 \sinh \left({\frac{\pi\omega_k}{a}}\right)}} B(p_u)\\
\hat{h_k}^{(v)}(p_v) &= \frac{\Gamma\left(\frac{i\omega_k}{a} + 1\right)}{\sqrt{2 \sinh \left({\frac{\pi\omega_k}{a}}\right)}} B(p_u)
\end{split}
\end{equation}
\subsection{Interpretation}
The driving source $\rho$, with mixed absorption and emission thrusts $\alpha$ + $\beta$ = 1, contribute excitations to the scalar field $\phi$. Equations (\ref{ab}) and (\ref{fourier}) let us write down the expected change of energy
\begin{equation}  
  \label{number}
  \begin{split}
    \mathbb{E}[\Delta E] &= \frac{1}{4\pi} \int{|\rho(p)|^2 dp} \\
    &= \frac{\alpha}{4\pi} \int{\left|\hat{h_k}^{(u)}(p_u)\right|^2 dp_u} + \frac{\beta}{4\pi}\int{\left|\hat{h_k}^{(v)}(p_v)\right|^2dp_v} \\
    &= \frac{\left|\Gamma\left(\frac{i\omega_k}{a} + 1\right)\right|^2}{2 \sinh \left({\frac{\pi\omega_k}{a}}\right)} \frac{1}{4\pi} \int{{\left|B(p)\right|^2} dp} \\
    &=  \frac{\left|\Gamma\left(\frac{i\omega_k}{a} + 1\right)\right|^2}{2 \pi a \sinh \left({\frac{\pi\omega_k}{a}}\right)} \int{a/|p|^2 dp}\\  
&=I(\omega_k) P
  \end{split}
\end{equation}
where the integrals are on the positive energy massless shell with contributions from $p_u$ on the left piece and $p_v$ on the right piece.  We factored out $P = \int{a/|p|^2}$ with a remaining $p$ independent positive real coefficient $I(\omega_k)$.

Without being more careful we end up with inferred problems --- The integrals do not converge at zero, where $P$ explodes.  But this infinity cancels when we compare the spectral radiances $I(\omega_k)$ to each other.

The magnitude of our Gamma function has known asymptotics \cite[Eq.~5.11.9]{NIST:DLMF}
\[
\left|\Gamma\left(\frac{i\omega_k}{a} + 1\right) \right|^2 \sim \left(\frac{2 \pi \omega_k} {a}\right) e^{-\frac{\pi\omega_k}{a}}
\]
Plugging this into equation (\ref{number}) we find the average energy of the mode, the (1,1) dimensional Planck distribution function, and thus recover the Unruh's radiation spectrum\footnote{Compare to equation (\ref{radeq}) and references \cite{unruh} and \cite{Frodden}.} from a thrust driven field.
\[
\frac{1}{P} \mathbb{E}[\Delta E] = I(\omega_k) \sim \frac{\omega_k}{e^{\frac{2 \pi \omega_k}{a}}-1}
\]

\section{Conclusion and Prediction}
If we do not account for the thrust required to accelerate a detector, then we recover it instead as a thermal unknown in the vacuum, Unruh radiation.  But, it would seem that we can explain Unruh radiation directly using thrust in the illustrative ways outlined in this paper.  The prediction of this note is that Unruh radiation and Hawking-Bekenstein radiation, separate from thrust, should not appear.

\section{Acknowledgments}
Thanks to Daniel Justice for useful discussions.

\bibliographystyle{ieeetr}
\bibliography{bibliography}

\end{document}
