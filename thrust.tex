\documentclass[12pt,a4paper]{article}
\usepackage[width=.75\textwidth]{caption}
\usepackage{graphicx}
\usepackage{authblk}
\usepackage{amsmath}
\usepackage{amsfonts}
\usepackage{braket}
\usepackage{epigraph}
%\usepackage{mathrsfs}
\usepackage[mathscr]{euscript}
\usepackage[top=2cm, bottom=2cm, left=2cm, right=2cm]{geometry}
\usepackage{fancyhdr}

\setlength{\epigraphwidth}{0.8\textwidth}

% \pagestyle{fancy}
\begin{document}

%title and author details
\title{Field Theoretic Perspective on Thrust of an Accelerating Frame}
\author[1]{Kevin Player\footnote{kplayer@andrew.cmu.edu}}

\maketitle

\epigraph{The Unruh effect tells us that what we call particles is really just a matter of perspective.}{Lee Smolin}

\abstract{We analyze the quantum field theoretic framework where Unruh radiation was first described, interpreting acceleration as a driving source influencing the field. By conceptualizing this source as thrust, we provide a more intuitive mechanism for understanding radiation in accelerating frames.  Using the equivalence principle, we extend this framework to Hawking radiation as well, demonstrating how a common source-response relationship manifests in accelerating frames.}


\section{Introduction}

The objective of this paper is to show how Unruh radiation can be interpreted a thrust. We begin in Section 2 with preliminaries, following \cite{Frodden}, including a review of Rindler coordinates, wave equations, and their solutions in 1+1-dimensional Minkowski and Rindler spacetimes. Section 3 introduces field-theoretic driving forces, following \cite{beisert}, and extends the formalism using Fourier transforms to analyze sources as a combination of emission and absorption terms. In Section 4, we reinterpret Unruh radiation as thrust. Finally, Section 5 discusses Hawking radiation through the lens of the equivalence principle."

Much of the content, including notation and conventions, is directly adapted from the overview in \cite{Frodden}, the foundational work in \cite{unruh}, and the detailed treatment in \cite{beisert}. Consistent with these references, our focus is restricted to the free scalar field, which suffices to elucidate the Unruh effect.

\section{Rindler and Minkowski Modes Review}

Let $\hbar$ = $c$ = 1 and consider a uniform linear acceleration in $1+1$ dimensional spacetime. The full $1+3$ dimensional case does not add anything to the discussion, so without loss of generality we stick to the dimensions time $t$ and space $x$ where the boost is taking place.  The massless Klein-Gordon equation 

\begin{equation}
  \Box \phi = \frac{\partial^2 \phi}{\partial t^2} - \frac{\partial^2 \phi}{\partial x^2} = 0,
 \label{massless-wave-eq}
\end{equation}
has solutions which we describe using a basis  $u = -t + x$ and $v = -t - x$, the null coordinates along the light cone.  Any function of purely $u$, $f(u)$, is made up of modes that are constant on $v$.  That is they are given by a 1-dimensional Fourier expansion
\begin{equation}
  f(u) = \frac{1}{2\pi} \int{e^{i p_u u} \hat{f}(p_u) dp_u}.
\end{equation}
supported on $p_v = 0$.  The same is true for $v$, $g(v)$, $p_v$, and $\hat{g}(v)$ supported on $p_u = 0$.  The full space of solutions to equation (\ref{massless-wave-eq}) are combinations of $f$ and $g$, linear combinations of $\hat{f}(p_u) \delta(p_v)$ and $\hat{g}(p_v) \delta(p_u)$.  These are supported on the massless shell $E = \pm p$.  A basis of solutions are the ``Minkowski modes''
\begin{equation}
  \varphi_k(x,t) = \frac{1}{\sqrt{4 \pi \omega_k}} e^{i(k x - \omega_k t)}
\end{equation}
along with their complex conjugates, where $\omega^2 = k^2$, covering both the positive and negative frequency cases $\omega = \pm k$.

\begin{figure}[h]
\centering
\includegraphics[scale=0.2]{rindler_w.png}
\caption{Rindler wedge $W$ on the right.}
\label{rindlerw}
\end{figure}

The following notations and results for the rest of this section are taken from \cite{Frodden}. Let $W$ be the $x>|t|$ Rindler ``wedge'' with coordinates
\begin{equation}
  t = \frac{1}{a}e^{a\xi}\sinh{(a\eta)}
\label{sinh}
\end{equation}
\begin{equation}
x = \frac{1}{a}e^{a\xi}\cosh{(a\eta)}
\end{equation}
where $a>0$ is an acceleration parameter (see Figure \ref{rindlerw}). These are hyperbolic polar coordinates with an exponential "radius".  The massless Klein-Gordon equation in Rindler coordinates is
\begin{equation}
  \Box \phi = e^{-2a \xi}(-\partial_\eta^2 + \partial_\xi^2) \phi = 0
\end{equation}
which has a basis of solutions 
\begin{equation}
 r_k(\eta,\xi) = \frac{1}{\sqrt{4 \pi \omega_k}} e^{-i(\omega_k \eta -k \xi)}
\end{equation}
(along with conjugates) for each wave number $k$ and positive frequency $\omega_k = |k|$.  These ``Rindler modes'' do not immediately extend to the entire $(x,t)$-plane but are confined to the Rindler wedge $W$. 

Consider the analytic functions on the entire $(x,t)$-plane
\begin{equation}
f_k(u) = \frac{e^{\frac{\pi \omega_k}{2a}} {(au)}^{\frac{i\omega_k}{a}}}{ \sqrt{2\sinh\left(\frac{\pi\omega_k}{a}\right)}}
\end{equation}
\begin{equation}
g_k(v) = \frac{e^{\frac{\pi \omega_k}{2a}} {(av)}^{\frac{i\omega_k}{a}}}{ \sqrt{2\sinh\left(\frac{\pi\omega_k}{a}\right)} }
\end{equation}
These two functions are functions of $u$ and $v$ respectively, and so are solutions to equation (\ref{massless-wave-eq}).  Each $r_k$ can be extended to the entire plane as
\begin{equation}
e^\frac{\pi\omega_k}{2a} f_k(u) - e^{-\frac{\pi\omega_k}{2a}} g_k(v)  = \sqrt{2 \sinh \left({\frac{\pi\omega_k}{a}}\right)} r_k(\eta,\xi)
\end{equation}
see \cite{Frodden}.

Let $c_k^{(1)}$ and $c_k^{(2)}$ be annihilation and creation operators for the positive and negative frequency Minkowski modes respectively.  Define
\begin{equation}
  b_k = \frac{1}{\sqrt{2 \sinh\left(\frac{\pi \omega_k}{a}\right)}} \left( e^{\frac{\pi\omega_k}{2a}} c_k^{(1)} + e^{-\frac{\pi\omega_k}{2a}} c_{-k}^{(2) \dagger} \right).
\end{equation}
Then the number of particles that an accelerating observer will see in the Minkowski vacuum $\ket{0_M}$ is seen to be 
\begin{equation}
\label{radeq}
\braket{n} = \bra{0_M} b_k^\dagger b_k \ket{0_M} = \frac{1}{e^{\frac{2 \pi \omega_k}{a}}-1}
\end{equation}
which is usually interpreted as (Unruh) radiation.

\section{Fourier Transform of the Sources}
We will take several Fourier transforms to study the various modes and set up for some
integrals.  Let $\phi$ be a free scalar field in the flat $1+1$ dimensional Minkowski spacetime.  We will consider $f_k(u)$ and $g_k(v)$ as driving $W$-event horizon sources
\begin{equation}
\label{ab}
\rho(u,v) = \alpha f_k(u) + \beta g_k(v).
\end{equation}
with non-negative real convex combination $\alpha + \beta = 1$.
The drivers $f_k(u)$ and $g_k(v)$ are functions $f_k(u)$ of the past $W$-horizon and $g_k(v)$ of the future $W$-horizon respectively.  Both generate excitations, which we identify with absorption and emission thrusts respectively.


The sources can originate from a coupling term, $\rho \phi$, added to the free scalar Lagrangian
\begin{equation}
\mathscr{L}_{driven} = \mathscr{L}_{free} + \rho\phi 
\end{equation}
where
\begin{equation}
  \mathscr{L}_{free} = -\frac{1}{2} \partial^\mu \phi \partial_\mu \phi.
\end{equation}
This leads to an inhomogeneous Klein-Gordon equation
\begin{equation}
\Box  \phi = \rho
\end{equation}
as presented in \cite{beisert}\footnote{In \cite{beisert} it is assumed that the source is only active for a finite amount of time.  We let $\rho$ be active for all time.  The argument in \cite{beisert} seem to be adaptable to $\rho$.}

We want to integrate $\rho$ on shell in momentum space, which for a massless source is the positive energy part of the massless shell.  The two positive energy ``horizons'' border $p_u \le 0$ and $p_v \le 0$, see Figure \ref{masslessshell}.  Proceeding to take the Fourier transform of the function $f(u)$\footnote{WLOG since $g(v)$ is of the same form.}, we drop $p_u$ to just $p$ for the time being to increase legibility.  We will continue to assume that $\omega_k$ and $a$ are positive. Define the kernel

\begin{equation}
  A = e^{-i p u} (au)^\frac{i\omega_k}{a} du
\end{equation}
and then we have
\begin{equation}
\label{finalnorm}
  \hat{f}(\_u) =  \frac{e^{\frac{\pi \omega_k}{2a}}}{\sqrt{2 \sinh \left({\frac{\pi\omega_k}{a}}\right)}}  \int_{-\infty}^\infty A
\end{equation}
where $L=\int_{-\infty}^0 A$ and $R=\int_0^\infty A$ are the left and right sides of the total integral $L + R$.

\begin{figure}[h]
\centering
\includegraphics[scale=0.5]{massless_shell.png}
\caption{Massless shell is when $p_u$ = $p_v$ = 0}
\label{masslessshell}
\end{figure}


We rewrite the integrals using a complex changes of variables, $s$ = $ipu$, and contour integrals.

\begin{figure}[h]
\centering
\includegraphics[scale=0.3]{contour.png}
\caption{Using contours with large radius we convert the $L$ integral that goes to $i\infty$, and the $R$ integral that goes to $-i\infty$, to integrals with $s$ going to real $\infty$.}
\label{fig:x cubed graph}
\end{figure}


The $L$ integral for real $p<0$ is
\begin{equation}
\begin{split}
  L(p) & = -\int_0^{i\infty} \left(\frac{ias}{-p}\right)^\frac{i\omega_k}{a} \left(\frac{i}{-p}\right)ds \\
  & = \frac{-i}{a} \left(\frac{a}{-p}\right)^{\frac{i\omega_k}{a} + 1} \int_0^\infty \left(is\right) ^ \frac{i\omega_k}{a} e^{-s} ds \\
  & = \frac{-i}{a} \left(\frac{a}{-p}\right)^{\frac{i\omega_k}{a} + 1} \Gamma\left(\frac{i\omega_k}{a} + 1\right) e^{-\frac{\pi \omega_k}{2a}} \\
  & = \frac{1}{2} \Gamma\left(\frac{i\omega_k}{a} + 1\right) e^{-\frac{\pi \omega_k}{2a}} B(p)\\
\end{split}
\end{equation}
where
\begin{equation}
B(p) = \frac{-2i}{a} \left(\frac{a}{-p}\right)^{\frac{i\omega_k}{a} + 1} 
\end{equation}
is as shown.  This is using a large radius contour which rotates the endpoint 90 degrees clockwise.

The same calculation for $R$ is done using a counter-clockwise contour this time.
\begin{equation}
\begin{split}
  R(p) = \frac{1}{2}\Gamma\left(\frac{i\omega_k}{a} + 1\right) e^{-\frac{\pi \omega_k}{2a}} B(p)
\end{split}
\end{equation}

We get back to $f_k(u)$ and apply the normalization from equation (\ref{finalnorm})
\begin{equation}
\widehat{f_k}(p_u) = \frac{e^{\frac{\pi \omega_k}{2a}}}{\sqrt{2 \sinh \left({\frac{\pi\omega_k}{a}}\right)}}  ( L(p_u) + R(p_u) )
\end{equation}
So
\begin{equation}
\label{fourier}
\begin{split}
\widehat{f_k}(p_u) & = \frac{\Gamma\left(\frac{i\omega_k}{a} + 1\right)}{\sqrt{2 \sinh \left({\frac{\pi\omega_k}{a}}\right)}} B(p_u)\\
\widehat{g_k}(p_v) &= \frac{\Gamma\left(\frac{i\omega_k}{a} + 1\right)}{\sqrt{2 \sinh \left({\frac{\pi\omega_k}{a}}\right)}} B(p_u)
\end{split}
\end{equation}
\section{Interpretation}
The driving source $\rho$, with mixed absorption and emission thrusts $\alpha$ + $\beta$ = 1, contribute excitations to the scalar field $\phi$. Equations (\ref{ab}) and (\ref{fourier}) let us write down the expected change of energy
\begin{equation}  
  \label{number}
  \begin{split}
    \mathbb{E}[\Delta E] &= \frac{1}{4\pi} \int{|\rho(p)|^2 dp} \\
    &= \frac{\alpha}{4\pi} \int{\left|\widehat{f_k}(p_u)\right|^2 dp_u} + \frac{\beta}{4\pi}\int{\left|\widehat{g_k}(p_v)\right|^2dp_v} \\
    &= \frac{\left|\Gamma\left(\frac{i\omega_k}{a} + 1\right)\right|^2}{2 \sinh \left({\frac{\pi\omega_k}{a}}\right)} \frac{1}{4\pi} \int{{\left|B(p)\right|^2} dp} \\
    &=  \frac{\left|\Gamma\left(\frac{i\omega_k}{a} + 1\right)\right|^2}{2 \pi a \sinh \left({\frac{\pi\omega_k}{a}}\right)} \int{a/|p|^2 dp}\\  
&=I(\omega_k) P
  \end{split}
\end{equation}
where the integrals are on the positive energy massless shell with contributions from $p_u$ on the left piece and $p_v$ on the right piece.  We factored out $P = \int{a/|p|^2}$ with a remaining $p$ independent positive real coefficient $I(\omega_k)$.

Without being more careful we end up with inferred problems --- The integrals do not converge at zero, where $P$ explodes.  But this infinity cancels when we compare the spectral radiances to each other, $I(\omega_{k_1}) / I(\omega_{k_2})$.

The magnitude of our Gamma function has known asymptotics \cite[Eq.~5.11.9]{NIST:DLMF}
\begin{equation}
\left|\Gamma\left(\frac{i\omega_k}{a} + 1\right) \right|^2 \sim \left(\frac{2 \pi \omega_k} {a}\right) e^{-\frac{\pi\omega_k}{a}}
\end{equation}
Plugging this into equation (\ref{number}) we find the average energy of the mode, the $1+1$ dimensional Planck distribution function, and thus recover the Unruh's radiation spectrum from a thrust driven field.
\begin{equation}
\frac{1}{P} \mathbb{E}[\Delta E] = I(\omega_k) \sim \frac{\omega_k}{e^{\frac{2 \pi \omega_k}{a}}-1}
\end{equation}
Compare this to equation (\ref{radeq}) and references \cite{unruh} and \cite{Frodden}.

\section{Conclusion and Prediction}
If the thrust required to accelerate a detector is not explicitly accounted for, it manifests instead as an apparent thermal feature of the vacuum—Unruh radiation. However, as demonstrated in this paper, Unruh radiation can be directly explained as a consequence of thrust. This perspective leads to the prediction that neither Unruh radiation nor Hawking-Bekenstein radiation should appear independently of the thrust that drives the system.

\begin{figure}[h]
\centering
\includegraphics[scale=0.5]{rocket.png}
\caption{Black hole radiating on the left, rocket thrusting on the right.}
\label{masslessshell}
\end{figure}

\section{Acknowledgments}
Thanks to Ben Commeau and Daniel Justice for useful discussions.

\bibliographystyle{ieeetr}
\bibliography{bibliography}

\end{document}
