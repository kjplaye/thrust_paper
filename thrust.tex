% TODOS
% Does J need to be normalized, wrt Rindler and Minkowski
% Work out details of Green's function
\documentclass[12pt,a4paper]{article}
\usepackage[width=.75\textwidth]{caption}
\usepackage{graphicx}
\usepackage{authblk}
\usepackage{amsmath}
\usepackage{amsfonts}
\usepackage{braket}
\usepackage{epigraph}
%\usepackage{mathrsfs}
\usepackage[mathscr]{euscript}
\usepackage[top=2cm, bottom=2cm, left=2cm, right=2cm]{geometry}
\usepackage{fancyhdr}
\newcommand{\dv}[1]{\mathrm{d} #1 \text{ }}
\newcommand*\diff{\mathop{}\!\mathrm{d}}
\newcommand\restr[2]{{% we make the whole thing an ordinary symbol
  \left.\kern-\nulldelimiterspace % automatically resize the bar with \right
  #1 % the function
  \vphantom{\big|} % pretend it's a little taller at normal size
  \right|_{#2} % this is the delimiter
  }}
\setlength{\epigraphwidth}{0.8\textwidth}

% \pagestyle{fancy}
\begin{document}

%title and author details
\title{Field Theoretic Thrust of an Accelerating Frame}
\author[1]{Kevin Player\footnote{kplaye@gmail.com}}

\maketitle

\epigraph{The Unruh effect tells us that what we call particles is really just a matter of perspective.}{Lee Smolin}

\abstract{We analyze the quantum field theoretic framework in which Unruh radiation arises, interpreting the radiation as sourced by a driving field. In the Rindler frame, these driving modes span an extended region of time and exhibit a red- and blue-shifted frequency spread due to their support over the full Rindler wedge, resulting in a thermalized response. To refine this picture, we construct an interpolation between these long-lived driving modes and localized wave packets with peaked Fourier spectra that do not display such frequency smearing. This allows us to reinterpret Unruh radiation in terms of thrust—a localized driving effect that excites the field without inducing a thermal response, thereby offering a complementary, nonthermal perspective on acceleration-induced radiation.}

\section{Introduction}
We first present some notation and review the Unruh effect in section 2.  We then introduce a source in section 3 that exactly captures the particle creation in the cross term of the Bogoliubov transform for Rindler modes extended to Minkowski spave.  In section 4,  we interpolate between the eternal Rindler mode source to a more localized wave packet version, and in section 5 we interpret the results.

\section{Unruh Effect Review and Notation}

We draw notation and standard results from Frodden and Vald{\'{e}}s \cite{Frodden}.

Let $\hbar$ = $c$ = 1. Consider a uniform linear acceleration in $1+1$ dimensional spacetime and use metric signature $\eta = (-1,1)$. The full $1+3$ dimensional case does not add anything to the discussion, so without loss of generality we stick to the dimensions time $t$ and space $x$ where the boost is taking place.  Let $\phi$ be a free massless scalar field.

Consider the free scalar massless Lagrangian
\begin{equation}
\mathscr{L}_{free} = -\frac{1}{2} \eta^{\mu\nu}\partial_\mu \phi \partial_\nu \phi.
\end{equation}
We consider positive frequency modes with dispersion relation $\omega_k = |k| > 0$ as solutions to the resulting Klein-Gordon equation 

\begin{equation}
  \Box \phi = -\frac{\partial^2 \phi}{\partial t^2} + \frac{\partial^2 \phi}{\partial x^2} = 0,
 \label{massless-wave-eq}
\end{equation}
where $\Box = \eta^{\mu\nu} \partial_\mu \partial_\nu$. We expand $\phi$ in terms of ladder operators $a_k, a_k^\dagger$

\begin{equation}
  \int \diff k \, a_k \varphi_k(x,t) + h.c.
\end{equation}
where

\begin{equation}
  \varphi(x,t) = \frac{1}{\sqrt{4\pi\omega_k}} e^{i(kx - \omega_k t)}.
\label{amode}
\end{equation}
are positive frequency modes normalized with respect to the Klien-Gordon inner product at any time slice, say $t = 0$,
\begin{equation}
  \left<f, g\right>_{KG} = i \int \diff x (f^* \partial_t g - \partial_t f^* g).
\end{equation}

\subsection{Rindler Coordinates}

Consider the Rindler ``wedge''
\begin{equation}
  W_c = \{(x,t) : x-c>|t|\}
\end{equation}
with apex at $(t,x)=(0,c)$.  We start with $W = W_0$ which is region $I$ pictured in Figure \ref{rindlerw})); with coordinates
\begin{equation}
  t = \frac{1}{a}e^{a\xi}\sinh{(a\eta)}
\label{sinh}
\end{equation}
\begin{equation}
x = \frac{1}{a}e^{a\xi}\cosh{(a\eta)}
\end{equation}
where $a>0$ is an acceleration parameter. These are hyperbolic polar coordinates with an exponential ``radius''.

\begin{figure}[h]
\centering
\includegraphics[scale=0.4]{rindler_w.png}
\caption{Rindler wedge $I$ on the right.}
\label{rindlerw}
\end{figure}

The massless Klein-Gordon equation in Rindler coordinates is
\begin{equation}
  \Box \phi = e^{-2a \xi}(-\partial_\eta^2 + \partial_\xi^2) \phi = 0
\end{equation}
which being the same form as equation (\ref{massless-wave-eq}), up to a conformal factor, has a basis of solutions with the same form as equation (\ref{amode})
\begin{equation}
 r_k(\eta,\xi) = \frac{1}{\sqrt{4 \pi \omega_k}} e^{-i(\omega_k \eta -k \xi)} + h.c.
\end{equation}
for each wave number $k$ and positive frequency $\omega_k = |k| > 0$.  These ``Rindler modes'' are in terms of $\eta$ and $\xi$ and are thus confined to the Rindler wedge $W$.  

\subsection{Unruh Modes}
For $\omega_k = k > 0$, and
\begin{equation}
  \begin{array}{lll}
    \alpha_k &= \frac{e^{\frac{\pi\omega_k}{2a}}}{\sqrt{2 \sinh \frac{\pi \omega_k}{a}}} &= \sqrt{\frac{1}{1 - e^{-2\pi\omega_k / a}}} \\
    \beta_k &= \frac{e^{\frac{-\pi\omega_k}{2a}}}{\sqrt{2 \sinh \frac{\pi \omega_k}{a}}} &= \sqrt{\frac{1}{e^{2\pi\omega_k / a} - 1}}, \\
  \end{array}
  \label{alpha_beta}
\end{equation}
noting that $\alpha_k^2 - \beta_k^2 = 1$. We analyticaly continue $r_k$, $r_{-k}$ to the $t,x$ plane as
\begin{equation}
  \begin{array}{ll}
    r_{+k} &= \frac{1}{\sqrt{4 \pi \omega_k}} e^{-i(\omega_k \eta - k \xi)} = \frac{1}{\sqrt{4 \pi \omega_k}} (a(-t + x + i \epsilon))^{\frac{i \omega_k}{a}} \\
    r_{-k} &= \frac{1}{\sqrt{4 \pi \omega_k}} e^{-i(\omega_k \eta + k \xi)} = \frac{1}{\sqrt{4 \pi \omega_k}} (a(-t - x + i \epsilon))^{\frac{-i \omega_k}{a}} \\
  \end{array}
\end{equation}
Define also
\begin{equation}
  \begin{array}{ll}
    \varphi_{+k} &= \frac{1}{\sqrt{4 \pi \omega_k}} e^{-i(\omega_k t - k x)}\\
    \varphi_{-k} &= \frac{1}{\sqrt{4 \pi \omega_k}} e^{-i(\omega_k t + k x)}\\
    \mu^R_k &= \alpha_k (r_k - r_{-k}^*) \\
    &= \frac{e^{\frac{\pi \omega_k}{2a}}}{\sqrt{2 \sinh \frac{\pi \omega_k}{a}}} \left(r_k - r_{-k}^* \right) \\
    &= \frac{1}{\sqrt{4 \pi \omega_k}\sqrt{2 \sinh \frac{\pi \omega_k}{a}}} \left( e^{\frac{\pi \omega_k}{2a}} \left(a(-t+x+i\epsilon)\right)^{\frac{i\omega_k}{a}} + e^{\frac{-\pi \omega_k}{2a}} \left(a(t+x-i\epsilon)\right)^{\frac{i\omega_k}{a}} \right) \\
    \mu^L_k &= \beta_k (r_k^* - r_{-k} )\\
    &=\frac{e^{\frac{-\pi \omega_k}{2a}}}{\sqrt{2 \sinh \frac{\pi \omega_k}{a}}} \left(r_k^* - r_{-k} \right) \\
    &=\frac{1}{\sqrt{4 \pi \omega_k}\sqrt{2 \sinh \frac{\pi \omega_k}{a}}} \left( e^{\frac{-\pi \omega_k}{2a}} \left(a(-t+x+i\epsilon)\right)^{\frac{-i\omega_k}{a}} + e^{\frac{\pi \omega_k}{2a}} \left(a(t+x-i\epsilon)\right)^{\frac{-i\omega_k}{a}} \right) \\
  \end{array}
\end{equation}
%TODO (verify U = r_k expressions)

\begin{figure}[h]
\centering
\includegraphics[scale=0.5]{unruh_mode_rainbow.png}
\caption{Space time diagrams for $\left[\begin{array}{ccc} \varphi_k & r_k & \mu^R_k \\ \varphi_{-k} & r_{-k} & \mu^L_k \end{array} \right]$.  Color is phase and brightness is magnitude. Rainbow order over time is consistent for $\varphi$ and $\mu$ since these are made up of positive frequency Minkowski modes. ($r_k$ is not consistent across the log branch.) The left moving Rindler modes $r_k$(top) correspond to emission and the right moving mode $r_{-k}$ (bottom) to absorption.}
\label{unruh_rainbow}
\end{figure}

\subsection{Bogoliubov Transforms}
We have a Bogoliubov transformation matrix from $M$ to $W_0$ of
\begin{equation}
  \left[ \begin{array}{l}
    a^{(0)}_k \\
    a^{(0)}_{-k} \\
    \hline
    a^{(0)\dagger}_k \\
    a^{(0)\dagger}_{-k} \\
 \end{array} \right] = 
  \left[
\begin{array}{rr|rr}
    \alpha_k &       0   &  0       & \beta_k \\
    0        & -\alpha_k & -\beta_k & 0 \\
    \hline
    0        & \beta_k   & \alpha_k & 0 \\
    -\beta_k &    0      &   0      & -\alpha_k \\
\end{array} \right]_{k,q}
\left[ \begin{array}{l}
    c^R_q \\
    c^L_q \\
    \hline
    c^{R\dagger}_q \\
    c^{L\dagger}_q \\
 \end{array} \right]
\end{equation}
for a change of basis from $a^{(M)}$ to $c^R$ and $c^L$
\begin{equation}
  \phi = \int \dv{q} \mu_q^R c_q^R + \mu_q^L c_q^L + h.c.
  \label{c_ladder}
\end{equation}
From this we compute the usual Unruh radiation equation with Planck spectrum (compare $\beta_k$ with equation (\ref{alpha_beta})) to obtain
\begin{equation}
  a_k^{(0)} = \alpha_k c_q^R + \beta_k c_q^{L\dagger}
\label{a_in_c}
\end{equation}

We compute the Bogoliubov coefficients for
\begin{equation}
  \begin{array}{ll}
  a^{(c)}_k &= \int \dv{q} \alpha^{(cM)}_{kq} a^{M}_q + \beta^{(cM)}_{kq} a^{(M)\dagger}_q \\
  &= \int \dv{q} \alpha^{(c0)}_{kq} a^{(0)}_q + \beta^{(c0)}_{kq} a^{(0)\dagger}_q
  \end{array}
\end{equation}
as
\begin{equation}
  \begin{array}{ccl}
    \alpha^{(cM)}_{kq} &= \left<\varphi_q, r_k^{(c)} \right> &= \frac{1}{a \pi} \sqrt{\frac{\omega_k}{\omega_q}} \left(\frac{a}{q}\right)^{\frac{i\omega_k}{a}} e^{\frac{\pi \omega_k}{a}} \Gamma\left(\frac{i\omega_k}{a}\right) \\
    \beta^{(cM)}_{kq} &= \left<\varphi_q^*, r_k^{(c)} \right> &= \frac{1}{a \pi} \sqrt{\frac{\omega_k}{\omega_q}} \left(\frac{a}{q}\right)^{\frac{-i\omega_k}{a}} e^{\frac{-\pi \omega_k}{a}} \Gamma\left(\frac{-i\omega_k}{a}\right) \\
    \alpha^{(c0)}_{kq} &= \left<r_q^{(0)}, r_k^{(c)} \right> &= \frac{1}{2 \pi a}\sqrt{\frac{\omega_k}{\omega_q}} (ac)^{\frac{-i(\omega_q - \omega_k)}{a}} B\left(\frac{i\omega_k}{a}, \frac{i(\omega_q - \omega_k)}{a}\right) \\
    \beta^{(c0)}_{kq} &= \left<r_q^{(0)*}, r_k^{(c)} \right> &= \frac{1}{2 \pi a}\sqrt{\frac{\omega_k}{\omega_q}} (ac)^{\frac{-i(\omega_q + \omega_k)}{a}} B\left(\frac{-i\omega_k}{a}, \frac{i(\omega_q + \omega_k)}{a}\right) \\

  \end{array}
\end{equation}
We can compare absolute magnitudes for $M$ v.s. $W_c$ and see that they don't depend on $q$ or $c$
\begin{equation}
  \begin{array}{cc}
    \left|\beta_{kq}^{(c_1M)}\right|^2 = \left|\beta_{kq}^{(c_2M)}\right|^2 & \\
    \left|\beta_{kq}^{(c_1M)}\right|^2 = \left|\beta_{kq}^{(c_2M)}\right|^2 & \\
    \left|\beta_{kq}^{(cM)}\right|^2 / \left|\alpha_{kq}^{(cM)}\right|^2 = e^{\frac{-2\pi\omega_k}{a}}  & \text{   (thermal factor)} \\
 \end{array}
\end{equation}
The $c$ independence is expected since Unruh radiation is translation invariant. The $q$ independence can be strengthened as the expected number of particles in mode $k$
\begin{equation}
 \int \dv{q} \left|\beta_{kq}^{(cM)}\right|^2 = \frac{e^{\frac{-2 \pi \omega_k}{a}}}{2 \sinh \frac{\pi \omega_k}{a}} \int \dv{q} \frac{2}{a\pi |q|}
\end{equation}
where we factor out the divergent part to recover the radiation equation again.

We next turn to $W_c$ v.s. $W_0$ and also find $c$ independence there 
\begin{equation}
  \begin{array}{c}
    \left|\alpha_{kq}^{(0c_1)}\right| = \left|\alpha_{kq}^{(0c_2)}\right| \vspace{4pt}\\
    \left|\beta_{kq}^{(0c_1)}\right| = \left|\beta_{kq}^{(0c_2)}\right| \\
  \end{array}
\end{equation}
\begin{equation}
  \begin{array}{ll}
      \left|\beta_{kq}^{(0c)}\right|^2 / \left|\alpha_{kq}^{(0c)}\right|^2 \vspace{4pt} &= \left|\Gamma\left(\frac{i(\omega_q + \omega_a)}{a}\right)\right|^2 / \left|\Gamma\left(\frac{i(\omega_q - \omega_a)}{a}\right)\right|^2 \vspace{4pt} \\
  & = \frac{(\omega_q - \omega_k) \sinh \pi (\omega_q - \omega_k)}{(\omega_q + \omega_k) \sinh \pi (\omega_q + \omega_k)} \\
  \end{array}
  \label{nonzero_ratio}
\end{equation}
which is somewhat more surprising since this implies that $\int \dv{q} \left|\beta_{kq}^{(c_2c_1)}\right|^2$ is a $c_1$ and $c_2$ independent factor  for every shifted wedge inclusion.  In other words, the expected number of particles for a mode $r^{(c_2)}_k$ of $W_{c_2}$ in $W_{c_1}$'s vacuum is independent of the choice of shift $c_2$ and $c_1$.

More explicitly we have a transform matrix of $\Lambda_c$ from $W_0$ to $W_c$ 
\begin{equation}
  \left[ \begin{array}{l}
    a^{(c)}_k \\
    a^{(c)}_{-k} \\
    \hline
    a^{(c)\dagger}_k \\
    a^{(c)\dagger}_{-k} \\
 \end{array} \right] = \underbrace{
  \left[
\begin{array}{rr|rr}
    A_c        &       0   &  B_c            &  0 \\
    0        &      -A_c   &  0            & -B_c \\
    \hline
    \overline{B_c}        &    0      &  \overline{A_c} & 0 \\
    0 &    -\overline{B_c}      &   0           & -\overline{A_c} \\
\end{array} \right]_{k,q} }_{\Lambda_c}
  \left[ \begin{array}{l}
    a^{(0)}_q \\
    a^{(0)}_{-q} \\
    \hline
    a^{(0)\dagger}_q \\
    a^{(0)\dagger}_{-q} \\
 \end{array} \right]
\end{equation}
where $A_c = \alpha_{kq}^{(c0)} = P_c A_1 P_c^{-1}$  and $B_c = \beta_{kq}^{(c0)} = P_c B_1 P_c$ for a diagonal phase factor matrix
\begin{equation}
  P_c = P_{c,rs} = \delta(r - s) c^{\frac{i\omega_r}{a}} = e^{\frac{i H}{a} \log c}
\end{equation}
We can write $\Lambda_c$ out compactly out as
\begin{equation}
  \Lambda_c = Q_c \Lambda_1 Q_c^{-1}
\end{equation}
where
\begin{equation}
  Q_c = \left[\begin{array}{cccc}
        P_c, & 0 & 0 & 0 \\
        0 & P_c & 0 & 0 \\
        0 & 0 & P_c^{-1} & 0 \\
        0 & 0 & 0 & P_c^{-1} \\
    \end{array} \right] 
\end{equation}
Note that $\lim_{c\to 0} \Lambda_c = 1$ since the limit of $\lim_{c\to 0} \alpha_{kq}^{(c0)} = 1$ and $\lim_{c\to 0} \beta_{kq}^{(c0)} = 0$, which corresponds nicely to $\lim_{c\to 0} W_c = W_0$. The composition of Bogoliubov transforms, $\Lambda_{nc} = \Lambda_c^n$, yields
\begin{equation}
  \begin{array}{ll}    
    Q_{nc} \Lambda_1 Q_{nc}^{-1}  &= \Lambda_{nc} \\
         &= \left(Q_c \Lambda_{c} Q_c\right) \left( Q_c^{-1} \Lambda_{c} Q_c\right) \cdots \left(Q_c \Lambda_{c} Q_c\right) \\
  &= Q_c \Lambda_c^n Q_c^{-1} \\
  \end{array}
\end{equation}
so that
\begin{equation}
  \begin{array}{ll}
  \Lambda_c^n &= Q_c^{-1} Q_{nc} \Lambda_1 Q_{nc}^{-1} Q{c} \\
  &= Q_n \Lambda_1 Q_n^{-1}
  \end{array}
\end{equation}
and more generally we have a one parameter group given by
\begin{equation}
  \left\{\Lambda_0^x = Q_x \Lambda_0 Q_x^{-1} : x \in \mathbb{R} \right\}.
\end{equation}
This is an explicit version of modular flow, due to the von Neumann algebra modular automorphism associated with the translation $W_0 \rightarrow W_c$, studied in detail in Tomita-Takesaki theory \cite{Borchers2000}.  There we find thermal KMS states between open set inclusions in a much more general setting.


Consider a sequence
\begin{equation}
  W_{c_n} \subseteq \cdots \subseteq W_{c_i} \subseteq \cdots \subseteq W_{c_j} \subseteq W_{c_2} \subseteq W_{c_1}
\end{equation}
Then each $W_{c_i} \subseteq W_{c_j}$ involves particle production with a fixed squared magnitude for mode $k$.  We calculate this expected number of $W_{c_i}$ particles for mode $k$ in $W_{c_j}$'s vacuum
\begin{equation}
  \bra{0_{W_{c_j}}} a^{(c_i)\dagger}_k a^{(c_i)}_k \ket{0_{W_{c_j}}} = \frac{1}{2 \pi^2 k \sinh \frac{\pi k}{a}} \int_{x=0}^\infty \frac{x \sinh x}{(x+\frac{\pi k}{a}) \sinh(x+\frac{\pi k}{a})}
\end{equation}
which diverges.  The integrand goes to $e^{-\frac{\pi k}{a}}$ as $x$ gets large, so we can see that the expected number of particles in ratio goes to
\begin{equation}
  \frac{1}{m (e^{2m} + 1)} = \frac{1}{k (e^{\frac{2\pi \omega_k}{a}} - 1)}.
\end{equation}

\section{Driving Sources}

We can ask, {\bf ``What exactly is accelerating the observer?''}. Up to now we have not specified any physical mechanism for the acceleration.  We also have not specified the location of the observer in the wedge, just that it exists in the accelerating frame.  And we also do not know where in the wedge the particles should appear. These unknowns reflect an effective lack of information about the observer’s precise location and the origin of excitations, a feature that underpins the thermal nature of the Unruh effect.

Figure \ref{emit_absorb} illustrates the situation: The Rindler mode $r_k$ corresponds to left-moving modes propagating toward the future horizon (interpreted as emission), and $r_{-k}$ to a right-moving modes originating from the past horizon (interpreted as absorption). The Rindler modes are constructed as superpositions of Minkowski modes, effectively smeared over a range of frequencies. This is visually evident in Figure \ref{unruh_rainbow}, where the local frequencies increase (blue-shift) near the horizons. This is made explicitly by the Bogoliubov coefficients $\alpha_{kq}^{(cM)}$ and $\beta_{kq}^{(cM)}$ which encode the Fourier decomposition of the Rindler modes through their Klein-Gordon inner products with the Minkowski modes $\varphi_q$ and $\varphi_q^*$, respectively. This frequency delocalization, tied to the observer's acceleration horizon, underlies the apparent thermal character of the radiation.

\begin{figure}[h]
\centering
\includegraphics[scale=1.5]{emit_absorb.png}
\caption{
The frequency $\omega_k$ of a Rindler mode appears green from the observer's point of view, representing a mid-spectrum excitation. The mode is depicted as striking a mirror at the tail end of a rocket, where it is reflected as a combination of absorption and emission processes. The emitted component is blue-shifted as it propagates toward the future horizon, while the absorbed component is red-shifted, having originated from the past horizon and traveled toward the observer.}
\label{emit_absorb}
\end{figure}


We offer a concrete answer to our initial question by introducing a driving source; a physical mechanism representing the thrust behind the acceleration. As we will see, this reframes the interpretation: the radiation term is no longer spontaneous, but instead effectively shifts back into the source that drives the motion.

A correlator is expressed as a time-ordered chain of creation and annihilation operators. Here, we aim to simulate the effect of a creation operator by introducing a source term $J(x)$ into the Lagrangian, which directly excites the field in a specific mode. Since $J(x)$ couples linearly, it prepares a coherent state that excites the chosen mode in a controlled, phase-coherent manner. To replicate the action of a creation operator, the source must be engineered such that its overlap with the mode functions $u_k(x)$ matches the operator’s action on the field.

The field can be expanded as in equation (\ref{c_ladder}), and the $\beta_k$-term in equation (\ref{a_in_c}) is responsible for the thermal particle content of the Minkowski vacuum as seen by Rindler observers. Without loss of generality\footnote{We only consider a fixed frequency and the $L$-mode, and extend the argument linearly.}, we simulate the effect of a creation operator $c_k^{L \dagger}$ using
\begin{equation}
  c _k^{L\dagger} = \left<\phi, \mu_k^{L*}\right> = \int \dv{x} \mu_k^{L*}(x) \phi(x)
\end{equation}
and the orthogonality of the mode functions in the Klein-Gordon inner product. In the generating functional formalism, setting $J(x) = -\beta_k u_k^{L*}$ the functional derivative $\frac{\delta}{\delta J}Z[J]|_{J=0}$ inserts $\phi$ into time-ordered correlators. Smearing this field insertion against $\mu_k^{L*}(x)^*$ thus projects onto $c_k^{L\dagger}$.

In Rindler coordinates, this source term prepares a modified field state in which the Rindler mode occupation differs from the thermal distribution of the Minkowski vacuum. Rather than simply adding energy, the source introduces a coherent excitation that cancels the mode structure induced by the Bogoliubov $\beta$-terms, effectively replacing their contribution. This lets us construct a state where the Rindler response is vacuum-like for mode k

\begin{equation}
  \bra{J}  b_k^{\dagger} b_k \ket{J} = 0
\end{equation}

\section{Appendix Formulas}
Some useful formula
\begin{equation}
  \int_{c}^\infty x^a (x-c)^b \dv{x} = c^{a+b+1} B(b+1, -a-b-1)
\end{equation}
TODO, check this next one
\begin{equation}
  \int_{-\infty}^\infty e^{ikx} x^b \dv{x} = -\frac{2i}{k^{b+1}} e^{\frac{-\pi b}{2}} \Gamma(b+1)
\end{equation}
This is useful for the shifted wedge inner products. Let
\begin{equation}
  f_{k,b,d} = \left(a\left(b(t-i\epsilon)+x\right)\right)^\frac{id\omega_k}{a}
\end{equation}
Then
\begin{equation}
  \left< f_{k,b_k,d_k}, f_{q,d_q,d_q}\right> = \frac{1}{2\pi} \sqrt{\frac{\omega_k}{\omega_q}} (ac)^\frac{i(d_k\omega_k - d_q \omega_q)}{a} \left((-d_k)\frac{b_k + b_q}{2} \right) B\left(\frac{id_k\omega_k}{a}, \frac{-i(d_k\omega_k - d_q \omega_q)}{a}\right)
\end{equation}
We frequently use
\begin{equation}
  |\Gamma(ib)|^2 = \frac{\pi}{b \sinh \pi b}
\end{equation}


\section{Conclusion and Prediction}
If the thrust required to accelerate a detector is not explicitly accounted for, it manifests instead as an apparent thermal feature of the vacuum—Unruh radiation. However, as demonstrated in this paper, Unruh radiation can be directly explained as a consequence of thrust. This perspective leads to the prediction that neither Unruh radiation nor Hawking-Bekenstein radiation should appear independently of the thrust that drives the system.

\begin{figure}[h]
\centering
\includegraphics[scale=0.5]{rocket.png}
\caption{Hawking picture of black hole radiating on the left. Our picture of a rocket thrusting on the right.}
\label{rocket}
\end{figure}

\section{Acknowledgments}
Thanks to Ben Commeau and Daniel Justice for useful discussions.

\bibliographystyle{ieeetr}
\bibliography{bibliography}

\end{document}
