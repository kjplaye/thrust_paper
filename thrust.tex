\documentclass[12pt,a4paper]{article}
\usepackage[width=.75\textwidth]{caption}
\usepackage{graphicx}
\usepackage{authblk}
\usepackage{amsmath}
\usepackage{amsfonts}
\usepackage{braket}
\usepackage{epigraph}
%\usepackage{mathrsfs}
\usepackage[mathscr]{euscript}
\usepackage[top=2cm, bottom=2cm, left=2cm, right=2cm]{geometry}
\usepackage{fancyhdr}
\newcommand{\dv}[1]{\mathrm{d} #1 \text{ }}
\newcommand*\diff{\mathop{}\!\mathrm{d}}
\newcommand\restr[2]{{% we make the whole thing an ordinary symbol
  \left.\kern-\nulldelimiterspace % automatically resize the bar with \right
  #1 % the function
  \vphantom{\big|} % pretend it's a little taller at normal size
  \right|_{#2} % this is the delimiter
  }}
\setlength{\epigraphwidth}{0.8\textwidth}

% \pagestyle{fancy}
\begin{document}

%title and author details
\title{Causing the Effect: Localized Driving Sources and the Unruh Phenomenon}
\author[1]{Kevin Player\footnote{kplayer@andrew.cmu.edu} \footnote{kjplaye@gmail.com}}

\maketitle

\epigraph{The Unruh effect tells us that what we call particles is really just a matter of perspective.}{Lee Smolin}

\abstract{We present a framework that connects the thermal Unruh effect to a localized, non-thermal excitation picture. Using modular automorphisms, we track mode localization across nested Rindler wedges and construct compact wave-packet approximations via parabolic cylinder functions. This yields a smooth transition from global Rindler modes to fully localized excitations, supporting a causal, source-based interpretation in which the Unruh response arises from thrust, a directed, localized energy input, rather than passive detection of pre-existing vacuum correlations.}

\section{Introduction}
The Unruh effect\cite{unruh1976notes} states that uniformly accelerated observers perceive the Minkowski vacuum as a thermal state with particles. In this work, we develop a complementary viewpoint by examining how localization affects the thermal character of the field. In Section 2, we review the Unruh effect, including the relevant mode expansions and Bogoliubov transformations. In Section 3, we apply a classical source construction to inject particles into the field.

Section 4 explores partial localization by considering sub-regions of the Rindler wedge related through space-like translations and reflections, transformations that correspond to modular automorphisms in the associated operator algebras. We then use parabolic cylinder functions to construct a smooth interpolation between eternal Rindler modes and fully localized modes. Finally, in Section 5, we interpret the implications of our construction.

These results suggest an complementary interpretation: The Unruh effect arises from localized, non-thermal excitations, akin to a thrust driving the field.

\section{Preliminaries}

We draw notation and standard results from Frodden and Vald{\'{e}}s \cite{frodden2018unruh}. Let $\hbar$ = $c$ = 1. We consider a uniformly accelerating observer in 1+1 dimensional Minkowski spacetime with metric signature $\eta=(-1,+1)$. The extension to 1+3 dimensions does not affect the key physics of the Unruh effect, so we restrict to the (t,x) plane where the boost is occurring.  We only consider free scalar fields in this note.

Consider the free scalar massless Lagrangian
\begin{equation}
\mathscr{L}_{free} = -\frac{1}{2} \eta^{\mu\nu}\partial_\mu \phi \partial_\nu \phi.
\end{equation}
We consider positive frequency modes with dispersion relation $\omega_k = |k| > 0$ as solutions to the resulting Klein-Gordon equation 

\begin{equation}
  \Box \phi = -\frac{\partial^2 \phi}{\partial t^2} + \frac{\partial^2 \phi}{\partial x^2} = 0,
 \label{massless-wave-eq}
\end{equation}
where $\Box = \eta^{\mu\nu} \partial_\mu \partial_\nu$. We expand $\phi$ in terms of ladder operators $a_k, a_k^\dagger$

\begin{equation}
  \phi(x,t) = \int \diff k \, a_k \varphi_k(x,t) + \text{h.c.}
\end{equation}
where

\begin{equation}
  \varphi(x,t) = \frac{1}{\sqrt{4\pi\omega_k}} e^{i(kx - \omega_k t)}.
\label{amode}
\end{equation}
are pure Minkowkski positive frequency waves normalized with respect to the Klein-Gordon inner product over a Cauchy surface $\Sigma$ (usually $t = 0$)
\begin{equation}
  \left<f, g\right>_{KG} = i \int_\Sigma \diff x (f^* \partial_t g - \partial_t f^* g).
\end{equation}

\subsection{Rindler Coordinates}

To describe the physics from the point of view of a uniformly accelerating observer, we introduce Rindler coordinates \cite{frodden2018unruh,rindler1966kruskal} covering a right wedge 
\begin{equation}
  W = \{(x,t) : x>|t|\}
\end{equation}
with apex at the origin, pictured\footnote{All diagrams follow the convention of $t$ increasing upward and $x$ increasing to the right.} in Figure \ref{rindlerw}; with coordinates
\begin{equation}
  t = \frac{1}{a}e^{a\xi}\sinh{(a\eta)}
\label{sinh}
\end{equation}
\begin{equation}
x = \frac{1}{a}e^{a\xi}\cosh{(a\eta)}
\end{equation}
The constant acceleration parameter $a$ is introduced explicitly to make the dependence of the Unruh temperature, $T = \frac{a}{2\pi}$, manifest in subsequent expressions. The coordinates ($\eta$, $\xi$) describe the proper time and position in the frame of a uniformly accelerating observer, with world-lines of constant $\xi$ corresponding to hyperbolic trajectories in Minkowski spacetime.

\begin{figure}[h]
\centering
\includegraphics[scale=0.2]{rindler_w.png}
\caption{Rindler wedge $W$ on the right.}
\label{rindlerw}
\end{figure}

The massless Klein-Gordon equation in Rindler coordinates is
\begin{equation}
  \Box \phi = e^{-2a \xi}(-\partial_\eta^2 + \partial_\xi^2) \phi = 0
\end{equation}
The wave equation retains the same structure as the Minkowski case, up to the overall conformal factor $e^{-2a\xi}$. Since this factor does not affect the null structure of the equation, the mode solutions retain the same plane wave form but in the Rindler coordinates
\begin{equation}
 r_k(\eta,\xi) = \frac{1}{\sqrt{4 \pi \omega_k}} e^{-i(\omega_k \eta -k \xi)} + \text{h.c.}
\end{equation}
for each wave number $k$ and positive frequency $\omega_k = |k| > 0$.  These ``Rindler modes'' are in terms of $\eta$ and $\xi$ and are thus confined to the Rindler wedge $W$.  Since Rindler coordinates only cover $W$ (the right wedge), these modes are not defined globally in Minkowski space.

\subsection{Unruh Modes}
To review how a uniformly accelerated observer perceives the Minkowski vacuum as a thermal bath, we construct the Unruh modes\cite{unruh1976notes}, analytic continuations of Rindler modes that are positive-frequency solutions with respect to Minkowski time\footnote{We mean that the modes contain no negative frequency Minkowski components.}. From now on let $\omega_k = k > 0$.

We define constants $\alpha_k$ and $\beta_k$ which satisfy $\alpha_k^2 - \beta_k^2 = 1$
\begin{equation}
  \begin{aligned}
    \alpha_k &= \frac{e^{\frac{\pi\omega_k}{2a}}}{\sqrt{2 \sinh \frac{\pi \omega_k}{a}}} = \sqrt{\frac{1}{1 - e^{-2\pi\omega_k / a}}}  \\
    \beta_k &= \frac{e^{\frac{-\pi\omega_k}{2a}}}{\sqrt{2 \sinh \frac{\pi \omega_k}{a}}} = \sqrt{\frac{1}{e^{2\pi\omega_k / a} - 1}} \quad \text{(thermal form)} \\
  \end{aligned}
  \label{alpha_beta}
\end{equation}
They show up throughout in mode normalizations\footnote{The normalizations come from computing the Klein Gordon inner product on Minkowski space and comparing it to inner products on $W$ and $\widetilde{W}$.}, inner products, and resulting Bogoliubov transforms. $\beta_k$ also describes particle creation in terms of $|\beta_k|^2$ which has a thermal character, matching a Planck distribution at temperature $T = \frac{a}{2\pi}$.

Let $\widetilde{W}$ be the left Rindler wedge\footnote{Coordinates on this wedge are $t = -\frac{1}{a}e^{a\delta} \sinh(a\gamma)$, $x = -\frac{1}{a}e^{a\delta} \cosh(a\gamma)$; and $l_{\pm k} = \frac{1}{\sqrt{4 \pi \omega_k}} e^{i\omega_k(\gamma \pm \delta)}$.}, $x < -|t|$ with Rindler modes $l_{\pm k}$.  We analytically continue\footnote{From the definitions and properties of $\sinh$ and $\cosh$, it follows that $a(\pm t + x) = e^{a(\pm \eta + \xi)}$ and then $r_{\pm k} = e^{\pm \frac{i \omega_k}{a}\log a(\mp t + x \pm i\epsilon)}$.  Similar statements hold for the left wedge.} the Rindler modes $r_k$, $r_{-k}$, $l_k$ and $l_{-k}$ into the $(t,x)$ plane, these are the Unruh modes
\begin{equation}
  \begin{array}{ll}
    \mu^R_{\pm k} = \frac{\alpha_k }{\sqrt{4 \pi \omega_k}} (a(\mp t + x \pm  i \epsilon))^{\pm \frac{i \omega_k}{a}} & \hspace{20pt}
       {\mu^R_{\pm k}}_{|_W} \rightarrow \alpha_k r_{\pm k} \\
    \mu^L_{\pm k} =  \frac{\alpha_k}{\sqrt{4 \pi \omega_k}} (a(\mp t - x \pm  i \epsilon))^{\pm \frac{i \omega_k}{a}} & \hspace{20pt}
    {\mu^L_{\pm k}}_{|_{\widetilde{W}}} \rightarrow \alpha_k l_{\pm k} \\
  \end{array}
\end{equation}
We added an $i \epsilon$ prescription to dictate which branch of the log to take so that the modes are analytic and bounded on the $\mathfrak{I}(t) < 0$ half plane.  This makes the modes positive-frequency with respect to $t$.  Another way of writing the Unruh modes is
\begin{equation}
\begin{aligned}
  \mu^R_{\pm k} &= \alpha_k r_{\pm k} + \beta_k l^*_{\mp k} \\
  \mu^L_{\pm k} &= \alpha_k l_{\pm k} + \beta_k r^*_{\mp k} \\
\end{aligned}
\label{unruh_mode_def}
\end{equation}
where the right and left modes ($r_{\pm k}$ and $l_{\pm k}$) are understood to be zero outside of their respective wedges. See Figure \ref{unruh_rainbow} for an illustration of the Unruh modes. The magnitude shown jumps across the branch cut and conjugates the phase.  Also, note that we have ``twice as many'' Unruh modes as we have Rindler modes; we double count each $r_k$ with two analytic extensions $\mu^R_k$ and $\mu^{L*}_{-k}$, and similarly for the left modes.

The Unruh modes form an alternative orthonormal basis of solutions to the Klein-Gordon equation, distinct from the plane waves $\varphi_{\pm k}$ see the original source Unruh\cite{unruh1976notes}. The Unruh modes diagonalize (we will see in equation (\ref{diag})) the Minkowski vacuum in terms of Rindler particle states and thus provide the natural framework for describing the Unruh effect and the thermal response perceived by uniformly accelerated observers.

\begin{figure}[h]
\centering
\includegraphics[scale=0.5]{unruh_mode_rainbow.png}
\captionsetup{width=0.7\textwidth}
\caption{Spacetime diagrams of the $k>0$ mode functions $\left[\begin{array}{ccc} \mu^L_{-k} & \varphi_k & \mu^R_k \\ \mu^L_{k} & \varphi_{-k} & \mu^R_{-k} \end{array} \right]$. Color encodes the phase; brightness indicates magnitude. The Unruh modes change magnitude and conjugate phase across the log branch due to the interpretation of $\log(-1 \pm i\epsilon)$. The left-moving modes (top) correspond to emission in Minkowski space; the right-moving modes (bottom) to absorption.}
\label{unruh_rainbow}
\end{figure}

\subsection{Bogoliubov Transforms}
We generalize the wedge $W$ to a translated wedge $W_c$ with apex $(0,c)$
\begin{equation}
 W_c = \{(t,x) : x - c > |t|\} 
\end{equation}
and a reflected (left) wedge $\widetilde{W}_c$ with apex $(0,c)$
\begin{equation}
 \widetilde{W}_c = \{(t,x) : x - c < -|t|\}.
\end{equation}
Let the superscripts $(0)$, $(c)$, $(\widetilde{c})$, and $(M)$ represent the $W_0$, $W_c$, $\widetilde{W}_c$, and Minkowski frames of reference respectively.  Let $(A \rightarrow B)$ represent an open set inclusion\footnote{In the algebraic formulation of QFT, spacetime regions correspond to operator algebras. Here, we adopt a complementary (though formally contravariant) perspective, whereby shifts in the wedge induce Bogoliubov transformations between operator algebras.} $A \subseteq B$.

This allows us to directly compute $W_0 \rightarrow M$ Bogoliubov coefficients from equation (\ref{unruh_mode_def}) for a change of basis from $a^{(M)}_q$ to $c^R_q$ and $c^L_q$
\begin{equation}
  \phi = \int \dv{q} \mu_q^R c_q^R + \mu_q^L c_q^L + \text{h.c.}
  \label{c_ladder}
\end{equation}
We find a 4 by 4 block matrix with blocks of the form
\begin{equation}
  \left[ \begin{array}{l}
      a^{(0)}_k \vspace{10 pt}\\
    a^{\widetilde{(0)}\dagger}_{-k} \\
 \end{array} \right] = 
  \left[
\begin{array}{cc}
    \alpha_k &      \beta_k \\
    \beta_k        & \alpha_k \\
\end{array} \right]
\left[ \begin{array}{l}
    c^R_k \\
    c^{L\dagger}_{-k} \\
  \end{array} \right]
\label{diag}
\end{equation}
and three others, where the $a^{\widetilde{(0)}\dagger}_{-k}$ is a left wedge annihilator. We can summarize the transform in the right wedge $W$ as
\begin{equation}
  a_k^{(0)} = \alpha_k c_k^R + \beta_k c_{-k}^{L\dagger}
\label{a_in_c}
\end{equation}
and three other similar relations for $a_{-k}^{(0)}$, $a_{k}^{(0)\dagger}$, and $a_{-k}^{(0)\dagger}$.

We next compute the more general non-diagonal mixed Bogoliubov transformations.
\begin{equation}
  \begin{array}{rll}
  (c \rightarrow M) : & a^{(c)}_k &= \int \dv{q} \alpha^{(c \rightarrow M)}_{kq} a^{M}_q + \beta^{(c \rightarrow M)}_{kq} a^{(M)\dagger}_q \\
  (c \rightarrow 0) : &   a^{(c)}_k &= \int \dv{q} \alpha^{(c \rightarrow 0)}_{kq} a^{(0)}_q + \beta^{(c \rightarrow 0)}_{kq} a^{(0)\dagger}_q \\
  (\widetilde{c} \rightarrow 0) : &   a^{(\tilde{c})}_k &= \int \dv{q} \alpha^{(\widetilde{c} \rightarrow 0)}_{kq} a^{(0)}_q + \beta^{(\widetilde{c} \rightarrow 0)}_{kq} a^{(0)\dagger}_q \\
  \end{array}
\end{equation}
We make use of a gamma function for $(c \rightarrow M)$. This occurs naturally in the KG dot product as an integral over an exponential phase from $\varphi_k$ and a $(x-c)$ power from $r_k^{(c)}$ (the Mellin transform of $e^{ikx}$ \cite{bracewell1966fourier}):
\begin{equation}
  \begin{array}{ccl}
    \alpha^{(c \rightarrow M)}_{kq} &= \left<\varphi_q, r_k^{(c)} \right> &= \frac{1}{2 \pi a} \sqrt{\frac{\omega_k}{\omega_q}} \left(\frac{a}{q}\right)^{\frac{i\omega_k}{a}} e^{\frac{\pi \omega_k}{2a}} \Gamma\left(\frac{i\omega_k}{a}\right) \\
    \beta^{(c \rightarrow M)}_{kq} &= \left<\varphi_q^*, r_k^{(c)} \right> &= \frac{1}{2 \pi a} \sqrt{\frac{\omega_k}{\omega_q}} \left(\frac{a}{q}\right)^{\frac{i\omega_k}{a}} e^{\frac{-\pi \omega_k}{2a}} \Gamma\left(\frac{i\omega_k}{a}\right) \\
  \end{array}
  \label{bogoCM}
\end{equation}
Next we consider products of shifted powers to go after $(c \rightarrow 0)$. We make use of a beta function for $(c \rightarrow 0)$ which occurs naturally in the KG dot product as an integral over a power of $x$ and of $x-c$, from $r_k^{(0)}$ and $r_k^{(c)}$ respectively.  We compute the Bogoliubov coefficients as
\begin{equation}
  \begin{aligned}
    \alpha^{(c \rightarrow 0)}_{kq} &= \left<r_q^{(0)}, r_k^{(c)} \right> = \frac{1}{2 \pi a}\sqrt{\frac{\omega_k}{\omega_q}} (ac)^{\frac{i(\omega_k - \omega_q)}{a}} B\left(\frac{i\omega_k}{a}, \frac{-i(\omega_k - \omega_q)}{a}\right) \\
    \beta^{(c \rightarrow 0)}_{kq} &= \left<r_q^{(0)*}, r_k^{(c)} \right> = \frac{1}{2 \pi a}\sqrt{\frac{\omega_k}{\omega_q}} (ac)^{\frac{i(\omega_k + \omega_q)}{a}} B\left(\frac{i\omega_k}{a}, \frac{-i(\omega_k + \omega_q)}{a}\right) \\
  \end{aligned}
  \label{bogoC0}
\end{equation}
The reflected diamond wedge version also yields a beta function, but with a different form
\begin{equation}
  \begin{aligned}
    \alpha^{(\widetilde{c} \rightarrow 0)}_{kq}     &= \left<r_q^{(0)*}, r_k^{(\widetilde{c})} \right> = \frac{1}{2 \pi a}\frac{\sqrt{\omega_k \omega_q}}{\omega_q - \omega_k} (ac)^{\frac{i(\omega_k - \omega_q)}{a}} B\left(\frac{i\omega_k}{a}, -\frac{i\omega_q}{a}\right) \\
    \beta^{(\widetilde{c} \rightarrow 0)}_{kq} &= \left<r_q^{(0)}, r_k^{(\widetilde{c})} \right> = \frac{1}{2 \pi a}\frac{\sqrt{\omega_k \omega_q}}{\omega_q + \omega_k} (ac)^{\frac{i(\omega_k + \omega_q)}{a}} B\left(\frac{i\omega_k}{a}, \frac{i\omega_q}{a}\right) \\
  \end{aligned}
  \label{bogoTC0}
\end{equation}

\subsection{Modular Automorphisms}

We can compare absolute magnitudes for $M$ v.s. $W_c$ and see that they don't depend on $c$
\begin{equation}
  \begin{array}{cc}
    \left|\alpha_{kq}^{(c_1 \rightarrow M)}\right|^2 = \left|\alpha_{kq}^{(c_2 \rightarrow M)}\right|^2 & \\
    \left|\beta_{kq}^{(c_1 \rightarrow M)}\right|^2 = \left|\beta_{kq}^{(c_2 \rightarrow M)}\right|^2 & \\
 \end{array}
\end{equation}
The $c$ independence is expected in this case since Unruh radiation is translation invariant. We next turn to $(c \rightarrow 0)$ and also find $c$ independence there 
\begin{equation}
  \begin{array}{c}
    \left|\alpha_{kq}^{(c_1 \rightarrow 0)}\right| = \left|\alpha_{kq}^{(c_2 \rightarrow 0)}\right| \vspace{4pt} \\
    \left|\beta_{kq}^{(c_1 \rightarrow 0)}\right| = \left|\beta_{kq}^{(c_2 \rightarrow 0)}\right| \\
  \end{array}
\end{equation}
This invariance is more surprising than in the Minkowski case, as it implies that the expected number of excitations for a mode $r_k^{(c_2)}$ when expressed in the vacuum of $W_{c_1}$,
\begin{equation}
  \int \dv{q} |\beta^{(c_2 \rightarrow c_1)}_{kq}|^2
\end{equation}
is invariant\footnote{Similar statements are true for reflected (diamond) wedges.} under changes in both $c_1$ and $c_2$.

More explicitly using the form of the $c$ term in equations (\ref{bogoC0}) and (\ref{bogoTC0}) we have a transform matrix of $\Lambda_c$ from $W_0$ to $W_c$ 
\begin{equation}
  \left[ \begin{array}{l}
    a^{(c)}_k \\
    a^{(c)}_{-k} \\
    \hline 
    a^{(c)\dagger}_k \\
    a^{(c)\dagger}_{-k} \\
 \end{array} \right] = \underbrace{
  \left[
\begin{array}{rr|rr}
    A_c        &       0   &  B_c            &  0 \\
    0        &      -A_c   &  0            & -B_c \\
    \hline 
    \overline{B_c}        &    0      &  \overline{A_c} & 0 \\
    0 &    -\overline{B_c}      &   0           & -\overline{A_c} \\
\end{array} \right]_{k,q} }_{\Lambda_c}
  \left[ \begin{array}{l}
    a^{(0)}_q \\
    a^{(0)}_{-q} \\
    \hline
    a^{(0)\dagger}_q \\
    a^{(0)\dagger}_{-q} \\
 \end{array} \right]
\end{equation}
where $A_c = \alpha_{kq}^{(c \rightarrow 0)} = P_c A_1 P_c^{-1}$  and $B_c = \beta_{kq}^{(c \rightarrow 0)} = P_c B_1 P_c$ for a diagonal phase factor matrix
\begin{equation}
  P_{c,rs} = \delta(r - s) c^{\frac{i\omega_r}{a}} = e^{\frac{i H}{a} \log c}
\end{equation}
where $H$ is the Rindler Hamiltonian associated with mode frequency $\omega_k$. We can write $\Lambda_c$ out compactly out as
\begin{equation}
  \Lambda_c = Q_c \Lambda_1 Q_c^{-1}
\end{equation}
where
\begin{equation}
  Q_c = \left[\begin{array}{cccc}
        P_c, & 0 & 0 & 0 \\
        0 & P_c & 0 & 0 \\
        0 & 0 & P_c^{-1} & 0 \\
        0 & 0 & 0 & P_c^{-1} \\
    \end{array} \right] 
\end{equation}
The composition of Bogoliubov transforms, $\Lambda_{nc} = \Lambda_c^n$, yields
\begin{equation}
  \begin{array}{ll}    
    Q_{nc} \Lambda_1 Q_{nc}^{-1}  &= \Lambda_{nc} \\
         &= \left(Q_c \Lambda_{c} Q_c\right) \left( Q_c^{-1} \Lambda_{c} Q_c\right) \cdots \left(Q_c \Lambda_{c} Q_c\right) \\
  &= Q_c \Lambda_c^n Q_c^{-1} \\
  \end{array}
\end{equation}
so that
\begin{equation}
  \begin{array}{ll}
  \Lambda_c^n &= Q_c^{-1} Q_{nc} \Lambda_1 Q_{nc}^{-1} Q{c} \\
  &= Q_n \Lambda_1 Q_n^{-1}
  \end{array}
\end{equation}
and more generally we have a one parameter unitary group under the modular parameter $x = \log c$, with generator $H/a$ given by
\begin{equation}
  \left\{\Lambda_1^x = Q_x \Lambda_1 Q_x^{-1} : x \in \mathbb{R} \right\}.
\end{equation}
So these Bogoliubov transformations between shifted wedges form a one-parameter group under translations of the apex, exhibiting a symmetry that parallels modular automorphism flow in algebraic QFT \cite{borchers2000revolutionizing}. In contrast to traditional treatments emphasizing Lorentz boosts within a fixed wedge, this formulation reveals modular structure via spatial translations.

Consider a sequence
\begin{equation}
  W_{c_n} \subseteq \cdots \subseteq W_{c_i} \subseteq \cdots \subseteq W_{c_j} \subseteq W_{c_2} \subseteq W_{c_1}
  \label{chain}
\end{equation}
The result is that each inclusion $W_{c_i} \subseteq W_{c_j}$ yields the same ``amount'' and ``type'' of particle production, with fixed squared Bogoliubov magnitude $|\beta_{kq}|^2$, implying that the expected number of particles remains constant across all nested wedge pairs, independent of the actual values of $c_i$ or $c_j$.  

\section{Driving Sources}

We now turn to a foundational question: {\bf ``What, physically, is accelerating the observer?''} In standard Unruh-effect derivations, the acceleration is treated as given; here we supply a concrete, physical mechanism in the form of a localized, entangled source.

Acceleration has been introduced as a geometric feature, a coordinate choice, without reference to any underlying dynamical mechanism. Moreover, we have left unspecified both the observer’s precise location within the Rindler wedge and the spatial origin of the detected excitations. These omissions reflect an effective coarse-graining over the details of the observer and their interaction with the field, a feature that contributes to the apparent thermality observed in the Unruh effect.

\begin{figure}[h]
\centering
\includegraphics[scale=1.0]{emit_absorb.png}
\captionsetup{width=0.7\textwidth}
\caption{A Rindler mode's frequency is smeared out in Minkowski space, blue-shifted near the horizon. We diagram a particle as if it were striking a mirror at the rear of a rocket, where its reflection emerges as a combination of emission and absorption processes in the Rindler frame.}
\label{emit_absorb}
\end{figure}

A natural physical interpretation is that a {\it driving source} must exist, both as the cause of the observer’s acceleration, and as a localized field source coupled to the quantum field. In this view, acceleration is not merely a kinematic artifact or coordinate reparameterization, but the result of active, localized interactions along the observer’s worldline. These interactions are responsible for the observer’s motion.  This plays well with the notion that the observer is not in an isotropic background radiation, but is actually thrusted away from the event horizon, the thrust being intimately correlated with the causal horizon.


Figure \ref{emit_absorb} illustrates the situation with a particle composed of Rindler modes. The modes $r_k$ are left-moving, propagating toward the future horizon and associated with {\bf emission}; the $r_{-k}$ modes are right-moving, originating from the past horizon and are associated with {\bf absorption}. These Rindler modes are constructed as superpositions of restricted Minkowski modes ${\varphi_q}_{|_W}$ , effectively smeared across a range of frequencies.  This frequency mixing is evident in Figure \ref{unruh_rainbow}, where the modes blue-shift infinitely near the horizons and red-shift infinitely at spatial infinity, due to the geometry of the wedge.

\subsection{Construction}

\begin{figure}[h]
\centering
\includegraphics[scale=0.75]{paz.png}
\captionsetup{width=0.7\textwidth}
\caption{A thruster in the past, Paz, emits an entangled photon pair that accelerates Wright (to the right) and Lester (to the left).}
\label{paz}
\end{figure}


We motivate a construction of a driving source using an example pictured in Figure \ref{paz}.  The observer on the right at $(t,x) = (0,1)$ is named Wright and we also have an observer on the left at $(0,-1)$ named Lester.  A common causal ancestor exists in the past, Paz at $(-1,0)$. Paz emits an entangled pair of photons with (Rindler) frequency $\omega_k$ as part of a stream of photons that are the source, and cause, of the acceleration of Wright and of Lester.  The accelerations are thus correlated.

Now where we would have usually created a localized detector for Wright, such as an Unruh-DeWitt detector \cite{unruh1976notes} \cite{einstein1979general}, and coupled it with the field, we instead seek to model the situation where Paz, a thruster, is coupled to the field. This inversion of effect and cause, of detector and thruster, is at the heart of our construction. Paz's emission makes the source of the acceleration manifest as particle injection, so that Wright will make sense of a guaranteed observation of a ``thrust particle'' with fixed peaked momentum $\omega_k$. Wright’s detection is entangled with Lester’s, so Lester will also observe a particle with momentum $\omega_k$, measurement induced correlation.

Physically, Paz introduces a bilocal source $J$
\begin{equation}
 \begin{array}{c}
 J(x,y) = \lambda f_L(x) f_R(y)  \\
 \mathscr{L}_{sourced} = \mathscr{L}_{free} + J\phi
 \end{array}
\end{equation}
This source construction is standard to QFT \cite{Schwinger_1966} \cite{ryder1996quantum} as a coupling of a classical function with the field as an interaction term $J\phi$. In this version $J$ acts coherently on the field modes in both wedges simultaneously, creating correlations that build entanglement at the level of the global quantum state. The product structure reflects this paired excitation, with $f_L$ and $f_R$ defining the mode profiles localized respectively in the left and right Rindler wedges. This is a paired global-coherent version of the usual vacuum squeezing.

To analyze a mode $f_R$ in the right wedge, we pair the mode with a complementary mode in the left wedge $f_L$.  This setup is fully compatible with the standard thermal vacuum description\footnote{The form of this vacuum equation can be proven by recursively applying the Bogoliubov relations in equation (\ref{a_in_c}). $\theta_k$ is chosen to match $\alpha_k = \cosh{\theta_k}$ and $\beta_k = \sinh{\theta_k}$ from equation (\ref{alpha_beta}) so $\tanh{\theta_k} = e^{-2\pi\omega_k / a}$.} of the Unruh effect
\begin{equation}
  \ket{0}_M = \prod_{\omega} \frac{1}{\cosh{\theta_k}} \sum_n \tanh^n{\theta_k} \ket{n_\omega}_{L} \ket{n_\omega}_{R}.
  \label{vacuum}
\end{equation}
The state prepared by Paz corresponds to a selective excitation with $n=1$ at a fixed $\omega$,
\begin{equation}
  \ket{\Psi_\omega} = \ket{1_\omega}_L \ket{1_\omega}_R,
\end{equation}
a controlled realization of one term in the ensemble. Our source singles out the $n_\omega=1$ term, giving a controlled ``slice'' of the full thermal ensemble. The source $J(x,y)$ does not contradict the thermal interpretation, but rather gives us a handle on one global microscopic element of the local right wedge thermal structure that we get by tracing over the left wedge.

\begin{figure}[h]
\centering
\includegraphics[scale=0.75]{paz_freq.png}
\hspace{20 pt}
\includegraphics[scale=0.235]{paz_bw.png}
\captionsetup{width=0.7\textwidth}
\caption{The points marked with black and white dots at (0,1) (Wright) and (-1,0) (Paz) respectively in the left picture lie on the same Unruh mode trajectory; although the mode is infinitely blueshifted at the horizon, the total Doppler effect between these two locations cancels, so the mode represents the same local frequency $\omega_k$. The right figure illustrates a finite time interval of width $\delta$ for the source $J$ to be active.}
\label{paz_freq}
\end{figure}

This setup is fully compatible with the standard thermal vacuum description of the Unruh effect. In the usual formulation, the Minkowski vacuum can be written as a thermal distribution of Rindler excitations, with each frequency mode $\omega_k$ described by a superposition over occupation numbers. Our construction simply selects the $n_\omega = 1$ sector, a single correlated excitation in each wedge, rather than the full thermal sum. In this sense, the process we describe is a specific ``microstate'' consistent with the broader thermal ensemble, corresponding to the case where exactly one entangled Rindler pair of frequency is present.

An illustration of this setup is shown in Figure \ref{rocket_inertial}. From this perspective, the apparent thermality arises from intrinsic properties of the vacuum, an effective ignorance of the source's detailed structure and dynamics. In this view, the Unruh effect is not a passive revelation of hidden particles in the vacuum, but a measurable consequence of {\it thrust}.


\begin{figure}[h]
\centering
\includegraphics[scale=0.5]{rocket_inertial.png}
\caption{Conceptual illustration of thermal v.s. localized acceleration.}
\label{rocket_inertial}
\end{figure}

\subsection{Physical Details}

The construction of the absorption mode $f_R$ for Wright presents a couple of notable physical challenges.  The first of these is that $J$ is described in the Rindler basis, so Paz will need to couple to an Unruh mode -- the analytically continued Rindler mode (see Section 2). The Rindler mode undergoes an infinite blue shift as it approaches the horizon separating the right and past wedges, but by analytically continuing it as an Unruh mode into the past wedge, we find a corresponding red shift when moving from the horizon to Paz. Figure \ref{paz_freq} (left picture) shows that the points at $(0,1)$ and $(-1,0)$ (Wright and Paz) experience the same local frequency, reflecting this symmetry.  

Although a single Minkowski photon is not a pure Unruh mode, the analytic continuation inherent in the Unruh basis captures the geometric frequency relationship between Wright and Paz.  A more general construction would place Wright and Paz in a more arbitraty configuration in which the local Fourier structure would not match exactly, there would be an overall finite Doppler shift involved.

In subsequent sections, we aim to localize $f_R$ away from the high-frequency, near-horizon behavior typical of thermal modes, making the physical realization of the source $J$ more plausible as a localized wave packet. For the remainder of this note, we focus on the modes $f_R$ in the right wedge without explicit reference to $f_L$, operating under the assumption that the localized driving source in the right wedge arises from a globally entangled paired source.

The second challenge is that we are integrating over the eternal wedge.  This challenge is met by noticing that our eternal modes $f_R$ and $f_L$ can be replaced by a finite time interval $|t + 1| < \delta$ around Paz's time $t=-1$ using Stokes' theorem (see Figure \ref{paz_freq} (right picture)). The coupling constant $\lambda$ multiplied by $\delta$ is choosen to match the vacuum equation (\ref{vacuum}) at $n_\omega = 1$.

Finally, we emphasize that our running example represents just one of many possible mechanisms for exchanging energy and correlations between wedges. The presence of virtual particle-antiparticle pairs in massive scalar fields further enriches this picture, though this extension lies beyond the scope of the present note.

\subsection{Back in Time Story}

An interesting side note is that Lester's proper time is flowing in reverse to Minkowski time (and to Wright's time). (Left wedge Rindler time runs in the opposite direction to Minkowski time.) So we can actually track the story\footnote{We can interchange these emission and absorption stories, Wright emits and Lester absorbs, but we would need to involve an observer in the future at $(t,x) = (1,0)$ for that.} as:
\begin{enumerate}
\item Lester emits a particle traveling back in time.
\item Paz receives this particle and reflects it forward in time to Wright.
\item Wright then absorbs the particle.
\end{enumerate}

\subsection{Sub-Wedge Causal Chain}

The sub wedge sequence mentioned in equation (\ref{chain}) has an interesting interpretation.  The vacuums are entangle from each wedge to the next. In our in construction, this is understood as a causal chain from Paz to Wright.  Each sub wedge inherits a source $J$ from the previous.

\section{Localization}
\subsection{Localization via Translated Wedge Inclusion}

Consider the two nested Rindler wedges $W_c \subseteq W_0$ shown in Figure \ref{restrict}. Let $r_q$ denote a Rindler mode\footnote{From here on we often interchange Rindler and Unruh modes since we are only concerned with the restriction to the right wedge or subsets.} associated with $W_0$, analytically continued to the entire Minkowski space. The gray-scale region indicates the full support of $r_q$, while the rainbow-colored segment shows its restriction to the sub-wedge $W_c$.

\begin{figure}[h]
  \centering
\includegraphics[scale=0.4]{wedge_in_wedge.png}
\caption{A Wedge $W_c$ (blue) inside of the wedge $W_0$ (white). Rindler mode $r_q$ of $W_0$ (gray-scale) restricted to $W_c$ (rainbow).}
\label{restrict}
\end{figure}

By considering the restriction of $r_q$ to $W_c$, we have partially localized the observer and the mode. The restriction effectively cuts off the high-frequency content of $r_q$ near the future horizon\footnote{Similarly, $r_{-q}$ experiences suppression near the past horizon.} of $W_0$. The resulting mode still spans the full spatial extent of $W_c$, but it is now insulated from the highly oscillatory behavior near the horizons of $W_0$.  The localization is not complete however, the observer can still be anywhere within the wedge $W_c$, and the corresponding modes $r_k$ still exhibit thermal characteristics because of low-frequency oscillations extending throughout the wedge.

To further study the situation, consider the modulus squared inner product $\left|\left<r_q^{(0)}, r_k^{(c)} \right>\right|^2$, also known as the Bogoliubov $\left|\alpha^{(c \rightarrow 0)}_{kq}\right|^2$, from equation (\ref{bogoC0}).  We fix $q$ and use $|\Gamma(ib)|^2 = \frac{\pi}{b \sinh \pi b}$ to obtain
\begin{equation}
  \left|\left<r_q^{(0)}, r_k^{(c)} \right>\right|^2 = \frac{\sinh \frac{\pi \omega_q}{a}}{4\pi a (\omega_q - \omega_k) \sinh \pi \frac{\omega_q - \omega_k}{a} \sinh \frac{\pi \omega_k}{a}}
\end{equation}
as a function of $\omega_k$. The function exhibits a second-order pole at $\omega_k = \omega_q$, resulting in a sharply peaked feature, see Figure $\ref{peaked}$. Although the $\sinh$ terms encode aspects of the familiar thermal distribution, especially broadening near $\omega_k = 0$, the existence of the peak itself at $\omega_k = \omega_q$ originates from the geometric restriction, not from a detector’s passive response to the vacuum, but as the active spectral footprint of a localized source.

\begin{figure}[h]
  \centering
\includegraphics[scale=0.6]{peaked.png}
\caption{The Rindler modes $r_k$ of $W_c$ show a peaked spectral overlap with $r_q$ at $\omega_k = \omega_q$. The incomplete beta function picks out the spectral peak suppressing the peak at zero.}
\label{peaked}
\end{figure}

\subsection{Diamond Localization via Reflected Wedge Intersection}

Further localization is achieved by intersecting $W_c$ with a reflected wedge $\widetilde{W}_{2c}$. This defines a more tightly localized diamond-shaped region, as shown in Figure \ref{diamond}. The mode $r_q$ is now restricted to the intersection $W_c \cap \widetilde{W}_{2c}$, which eliminates much of the infrared behavior previously associated with the unrestricted wedge.

The Klein Gordon inner product at $t=0$ now takes the form of an incomplete version of the beta function from equation (\ref{bogoC0}), corresponding to an integral\footnote{We could also use the other form of the beta function in equation (\ref{bogoTC0}) to compute the same inner product.} evaluated from $c$ to $2c$ rather than extending to infinity. This inner product does not however correspond to a mode expansion of the field, since the analytic continuation of the diamond would be to all of $W_c$, or all of $\widetilde{W}_{2c}$, in which case the mode would no longer be localized to the diamond.  So it is no longer invariant to integrate along a Cauchy surface for the inner product; this construction does not define a complete orthonormal set, and cannot be used to build a full basis of field modes.

This motivates a shift in perspective: rather than interpreting $r_q$ as part of a global mode expansion, we regard it as a compactly supported, non-invariant test function, i.e., a driving source localized to the diamond region, consistent with the source framework introduced in Section 3. We turn on $r_q$ exactly for a fixed period of $x-t$ (or $x+t$ for $r_{-q}$). The resulting spectral response in the diamond, computed from this truncated integral, is shown\footnote{The green (complete) beta function curve is actually independent of the choice of translation $c$, it is the same curve for any sub-wedge space-like inclusion (see Modular Automorphism Section 2.4). In contrast, the incomplete beta function curve (magenta) does depend on the endpoint ($2c$ is shown).} in Figure \ref{peaked}.


The plot reveals that the main spectral peak at $\omega_k = \omega_q$ persists, while the thermal contribution near $\omega = 0$ is significantly attenuated.

\begin{figure}[h]
  \centering
\includegraphics[scale=0.4]{diamond_in_wedge.png}
\caption{The same situation as in Figure \ref{restrict} but we further intersect with a reflected (left) wedge $\widetilde{W}_{2c}$ (green). Rindler mode $r_q$ of $W_0$ (gray-scale) restricted to $W_c \cap \widetilde{W}_{2c}$ (rainbow).}
\label{diamond}
\end{figure}


\subsection{Thermal to Localized Interpolation}
To further probe how global thermal structure transitions into a localized excitation profile, we consider the behavior of Rindler-to-Minkowski Bogoliubov coefficients when weighted by a Gaussian envelope. This allows us to interpolate between de-localized (thermal) and localized (spectrally peaked) behavior. We use parabolic cylinder functions \cite{AbramowitzStegun1964,Olver1959UniformAE} which are the analytic continuation of 

\begin{equation}
D_\nu(-z) = \frac{e^{-\frac{1}{4}z^2}}{\Gamma\left(-\nu\right)} \int_0^\infty  \dv{s} e^{zs} s^{-\nu - 1} e^{-\frac{1}{2} s^2}, \mathfrak{R}\nu < 0,
\end{equation}
where we use $-z$ instead of the usual $z$ so that future equations become simpler.


Without loss of generality, let $N_{\mu, \sigma} = e^{-\frac{1}{2} \frac{(x-t-\mu)^2}{\sigma^2}}$ be a (left-moving) Gaussian kernel with fixed $\mu$. We will multiply $\varphi_q^*$ by $N_{\mu, \sigma}$, but we could just as easily multiply $r_k$ by $N_{\mu, \sigma}$ for the same effect. A key aspect of this construction is that the resulting Minkowski modes are treated as driving sources rather than elements of an orthonormal mode expansion. Since we are not working within an orthonormal mode expansion, the normalization of $N_{\mu, \sigma}$ is left implicit. For clarity, since this setup may be non-standard, we carry out the calculations explicitly in this section.  We will examine the $N_{\mu,\sigma}$ modification of $\beta^{(c \rightarrow M)}_{kq}$ in  equation (\ref{bogoCM})

\begin{equation}
  \begin{aligned}
    \left< \varphi_q^* N_{\mu,\sigma}, r_k\right> &= \frac{1}{4\pi \sqrt{\omega_q \omega_k}} 2i \int_{\Sigma_W} e^{-i(\omega_q t - q x)} e^{-\frac{1}{2} \frac{(x-t-\mu)^2}{\sigma^2}} \partial_t (a(x-t))^\frac{i\omega_k}{a} \\
    &= \frac{1}{2\pi} \sqrt{\frac{\omega_k}{\omega_q}} a^{\frac{i \omega_k}{a} - 1} \int_0^\infty  \dv{x} e^{-\frac{1}{2} \frac{(x-\mu)^2}{\sigma^2} + i q x} x^{\frac{i\omega_k}{a} - 1} \\
    &= \frac{1}{2\pi} \sqrt{\frac{\omega_k}{\omega_q}} a^{\frac{i \omega_k}{a} - 1} \int_0^\infty \dv{x} e^{\left(-\frac{1}{2\sigma^2}\right) x^2 + \left(\frac{\mu}{\sigma^2} + i q \right) x + \left( -\frac{\mu^2}{2\sigma^2}\right)} x^{\frac{i\omega_k}{a} - 1}  \\
    &= \frac{1}{2\pi a} \sqrt{\frac{\omega_k}{\omega_q}} e^{-\frac{\mu^2}{2 \sigma^2}} \sigma^{\frac{i\omega_k}{a}} a^{\frac{i \omega_k}{a} }  \int_0^\infty \dv{s} e^{(\frac{\mu}{\sigma} + i q \sigma)s} s^{\frac{i\omega_k}{a} - 1} e^{-\frac{1}{2} s^2} \\
    &= \frac{1}{2\pi a} \sqrt{\frac{\omega_k}{\omega_q}} e^{-\frac{\mu^2}{2 \sigma^2}} {(\sigma a)}^{\frac{i \omega_k}{a}} e^{\frac{1}{4}(i q \sigma + \frac{\mu}{\sigma})^2} \Gamma\left(\frac{i\omega_k}{a}\right) D_{-\frac{i\omega_k}{a}}(-i q\sigma - \frac{\mu}{\sigma}) \\
    &=  \frac{1}{2\pi a } \sqrt{\frac{\omega_k}{\omega_q}} e^{-\frac{\mu^2}{2 \sigma^2}}  (\sigma a)^\frac{i\omega_k}{a} e^{\frac{1}{4} z^2} \Gamma(-\nu) D_\nu(-z)
  \end{aligned}
\end{equation}
where $\Sigma_W$ is the Cauchy surface $\eta=0$ on the Rindler wedge $W$,  $x = \sigma s$, $z = i q \sigma + \frac{\mu}{\sigma}$, and $\nu = -\frac{i \omega_k}{a}$. And then
\begin{equation}
  \begin{aligned}
    \left|\left< \varphi^*_q N_{\mu,\sigma}, r_k \right>\right|^2 &= \frac{1}{2\pi a \omega_q} \frac{\omega_k}{2\pi a} e^{-\frac{\mu^2}{\sigma^2}} \left| e^{\frac{1}{2} z^2} \right| \left| \Gamma(-\nu) \right|^2 \left| D_\nu(-z) \right|^2 \\
  \end{aligned}
\label{pcf}
\end{equation}
\begin{figure}[h]
\centering
\includegraphics[scale=0.5]{stokes.png}
\caption{Trajectory of $z = i q \sigma + \frac{\mu}{\sigma}$ as $\sigma$ interpolates between thermal and localized regimes. For $\sigma$ starting at $\infty$, the trajectory starts at the positive infinite imaginary axis, aligning with the dominant thermal component of the excitation. As $\sigma \to 0$, the system crosses a Stokes line and transitions into a sharply localized, source-driven configuration, where thermal character disappears. See also corresponding Figure \ref{pcf_sigma_curves}.}
\label{stokes}
\end{figure}
\begin{figure}
\centering
\includegraphics[scale=0.5]{pcf.png}
\caption{$\left|\left< \varphi^*_q N, r_k \right>\right|^2$ for various values of $\sigma$ (probability density is scaled for comparison). $\omega_q = 1$, $q = 1$, $a=1$, $\mu = 1$. See also corresponding Figure \ref{stokes}.}
\label{pcf_sigma_curves}
\end{figure}
From \cite{Olver1959UniformAE} we have
\begin{equation}
  D_\nu(-z) = e^{-i\pi\nu}z^{\nu}e^{-\frac{1}{4}z^2} \{ 1 + O(|z|^{-2}) \} + \frac{(2 \pi)^{\frac{1}{2}}}{\Gamma(-\nu)} z^{-\nu - 1} e^{\frac{1}{4}z^2} \{1 + O(|z|^{-2})\}
\label{asym}
\end{equation}
when $-\frac{1}{4}\pi + \epsilon \le \arg z \le \frac{3}{4} \pi - \epsilon$.

The two asymptotic regimes correspond to physically distinct interpretations: The second $e^{\frac{1}{4}z^2}$ term dominates for $z \to \infty$ as $\sigma \to 0$ and the first  $e^{-\frac{1}{4}z^2}$ term dominates for $z \to i \infty$ as $\sigma \to \infty$.  This is a Stokes phenomenon\footnote{See \cite{hashiba2021stokes} for a similar approach where the Stokes phenomenon is applied to particle production in simple expanding backgrounds, preheating after $R^2$ inflation, and a transition model with smoothly changing mass.} which flips over as we cross the Stokes line at $\arg z = \frac{\pi}{4}$. The situation is pictured in Figure \ref{stokes}.


We next combine equations (\ref{pcf}) and (\ref{asym}).  First for the thermal part that comes from the $e^{-\frac{1}{4}z^2}$ term where $\sigma \to \infty$ we have
\begin{equation}
  \begin{aligned}
    2\pi a \omega_q \left|\left< \varphi^*_q N, r_k \right>\right|^2 &= \frac{\omega_k}{2 \pi a} e^{-\frac{\mu^2}{\sigma^2}} \left|e^{-i\pi \nu} z^\nu\Gamma\left(\frac{i\omega_k}{a}\right)\right|^2   \\
    &= \frac{\omega_k}{2 \pi a} e^{-\frac{\mu^2}{\sigma^2}}  e^{\frac{-2\pi \omega_k}{a}} \left|e^{\frac{-2i\omega_k}{a} \log{(\frac{\mu}{\sigma} + iq\sigma)}}\right|^2  \frac{\pi}{\frac{\omega_k}{a} \sinh \frac{\pi \omega_k}{a}} \\
  &\to  e^{\frac{-2\pi \omega_k}{a}} \left|e^{\frac{-2i\omega_k}{a} \log{i}}\right|^2  \frac{1}{2\sinh \frac{\pi \omega_k}{a}} \\
  & =  e^{\frac{-2\pi \omega_k}{a}} e^{\frac{\pi \omega_k}{a}}  \frac{1}{ \left( e^{\frac{\pi \omega_k}{a}} - e^{\frac{-\pi \omega_k}{a}} \right)} \\
  & =  \frac{1}{e^{\frac{2\pi\omega_k}{a}} - 1} \\
  \end{aligned}
\end{equation}
which we expect by construction. For the localized part that comes from the $e^{\frac{1}{4}z^2}$ term where $\sigma \to 0$ we have
\begin{equation}
  \begin{aligned}
    2\pi a \omega_q \left|\left< \varphi^*_q N, r_k \right>\right|^2 &= \frac{\omega_k}{a} e^{-\frac{\mu^2}{\sigma^2}} \left|e^{z^2}z ^ {2(-\nu - 1)}\right| \\
    &= \frac{\omega_k}{a} e^{-\frac{\mu^2}{\sigma^2}} \left|e^{ \left(  iq\sigma + \frac{\mu}{\sigma} \right)^2 } e^{{2\left(\frac{i\omega_k}{a} - 1\right)\log{ \left(  iq\sigma + \frac{\mu}{\sigma} \right)   }   }   }\right| \\
    &\to \frac{\omega_k}{a} e^{\frac{-2\epsilon\omega_k}{a}} f(\sigma,\mu) \\
  \end{aligned}
\end{equation}
where the thermal pole at zero has disappeared. While the precise asymptotic form is not critical, we can control the ultraviolet behavior by taking $z$ to $(1+i\epsilon)\infty$ which remains within in the localized Stokes region. This introduces a regulating factor of the form $e^{-2\epsilon \omega_k / a}$, which suppresses high-frequency contributions. 

We should stop short of $\sigma$ actually reaching zero, since that is a localization to an extreme, where $N_{\mu,\sigma}$ shrinks to a bump with infinitesimal width; This is not a delta function, it is a vanishing source. We focus on the small-but-nonzero $\sigma$ regime where the thermal character has already disappeared. See Figure \ref{pcf_sigma_curves} for representative plots across varying values of $\sigma$.


\section{Conclusion}

This work examined the Unruh phenomenon from a localized perspective, emphasizing its manifestation as a physically driven effect rather than a purely thermal one. By restricting Rindler modes to translated and reflected wedges, and their intersections, we showed that the apparent thermal behavior can be partially eliminated. As these modes become localized, the traditional detector response, typically interpreted as a thermal signature of entanglement across a causal horizon, is here reinterpreted as the spectral imprint of a localized, dynamically sourced excitation. We then showed that the mixed Bogoliubov inner product between Minkowski and Rindler modes provides a smooth interpolation from global thermality to a localized spectral structure, with parabolic cylinder functions encoding this transition in analytic form.

Our approach complements the traditional detector-based interpretation by highlighting the role of the physical agent causing acceleration as an active source of field excitations. This perspective does not contradict the well-known thermality arising from entanglement but offers a localized, dynamical viewpoint that may better capture realistic scenarios.

\begin{figure}[h]
\centering
\includegraphics[scale=0.5]{rocket.png}
\caption{Conceptual illustration contrasting global Hawking radiation (left) with a localized, source-driven excitation near a black hole (right).}
\label{rocket}
\end{figure}

While our analysis is grounded in flat spacetime, the equivalence principle offers a natural pathway for extending this framework to curved geometries. In particular, future work could explore applications to Hawking radiation by modeling localized excitations near black hole horizons. Figure \ref{rocket} offers a schematic illustration of this idea, emphasizing a shift from thermal emission to localized, physically sourced excitations in the near-horizon region.

Finally, if the ideas presented here are correct, the thermal signature observed in the Unruh effect can always be traced back to a causal source. Experimentally, this implies that Unruh radiation should not appear isotropic, and randomly, but should be associated with specific, identifiable, accelerating ``thrust particles''. Conversely, the absence of a thermal Unruh signal in scenarios where a known accelerating source exists would provide experimental evidence supporting this framework. This question might already be settled if not for the intrinsic weakness of the Unruh effect.

\section{Acknowledgments}
I thank Frodden and Valdés for their excellent exposition \cite{frodden2018unruh}, and Beisert for his insightful lecture notes \cite{beisert2012quantum}. I'm also grateful to Ben Commeau, Daniel Justice, Edward Randtke, and ChatGPT for helpful discussions.


\bibliographystyle{ieeetr}
\bibliography{bibliography}

\end{document}
